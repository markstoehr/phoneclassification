% ---- ETD Document Class and Useful Packages ---- %
\documentclass{article}
\usepackage{amsfonts,bigints}
\usepackage{float}
\usepackage{natbib}
\usepackage{amsmath}
\usepackage{amssymb}
\usepackage{amsthm}
\usepackage{caption}
\usepackage{hyperref}
\usepackage{fullpage}
\usepackage{graphicx}
\usepackage{array}
\usepackage{dsfont}
\usepackage{subcaption}



\usepackage[ruled]{algorithm2e}
\renewcommand{\algorithmcfname}{ALGORITHM}
\SetAlFnt{\small}
\SetAlCapFnt{\small}
\SetAlCapNameFnt{\small}


%% Use these commands to set biographic information for the title page:
\title{Phone Classification Work}
\author{Mark Stoehr}
\date{\today}


\begin{document}

\maketitle
\tableofcontents

\section{Introduction}

Here we examine the performance of linear classifiers on a phone-classification task.  The setup is that we have
extracted all the labeled phones from the TIMIT database and we consider generative and discriminative classification
algorithms, as well as multiple different feature sets.  The primary feature set we consider are eight coarse binary edge features
(corresponding to four directions and two polarities) which are computed
at every time-frequency location of a time-frequency representation of the data.  The classification algorithms we employ
are the linear-kernel SVM and a product of Bernoullis mixture model likelihood ratio test.

\section{Experiments}

We describe experiments on the various feature types with the two different classifiers, the problem we are considering is a large
scale multiclass problem.

\subsection{Edges}

We work with discrete prolate spheroidal sequence-window multitaper spectrograms and then compute binary edge features on those.

\section{SVM on Edges}

We reduce the multiclass classification problem to $\binom{C}{2}$ binary classification problems and then we choose the class with a vote
by the different classifiers.  Formally we have phone labels $0,1,\ldots,C-1$
and for each pair $(i,j)$ where $i<j$ we have a linear classifier
$w_{i,j}$.  Our label prediction function is
\begin{equation*}\begin{array}{rcl}
f(X ; {w_{i,j}}_{i<j}) &=&\underset{i}{\arg\max} \sum_{j<i} \mathds{1}_{w_{j,i}^{\top}x > 0} + \sum_{i<j} \mathds{1}_{w_{i,j}^{\top}x < 0}\\
&=& \underset{i}{\arg\max} \; (\max\{\mathbf{0},(\operatorname{sgn} B_iWx)\})^\top \mathbf{1}
\end{array}\end{equation*}
where $W$ is the matrix whose rows are the linear classifiers $w_{i,j}$ and $B_i$ is a diagonal matrix whose entry 
$$B_i(k,k) = \begin{cases} 1 & W(k,:)^{\top} = w_{j,i}\;\text{for some } j\text{ so }j<i\\
                       -1 & W(k,:)^\top = w_{i,j}\;\text{for some } j\text{ so }i<j\\
                    0 & \text{otherwise}   \end{cases}$$
and $\operatorname{sgn}$ is a pointwise signum operator so that $\max\{\mathbf{0},(\operatorname{sgn} B_iWx)\}$ is a binary vector
whose number of ones is the number of linear classifiers that classify in favor of $i$.
The other main multiclass strategy would be train a 1-vs-rest classifier.

The performance obtained with \texttt{scripts/run\_svm\_exp.sh} included in the code directory.

\begin{table}[h]
  \centering
  \begin{tabular}{| l |  r |}
    \hline
     Penalty & Error Rate \\ \hline\hline
     0.1 & \input{exp/all_phones_exp/svm_little_reg_leehon_errorrate.txt} \\
     \hline
     0.01 & \input{exp/all_phones_exp/svm_reg_plus_leehon_errorrate.txt} \\
     \hline
     0.001 & \input{exp/all_phones_exp/svm_reg_plus_plus_leehon_errorrate.txt} \\
     \hline
     0.0001 &  \input{exp/all_phones_exp/svm_reg_plus_plus_plus_leehon_errorrate.txt} \\
     \hline
  \end{tabular}
  \caption{SVM Error rates for different penalties over edge features}
  \label{tab:myfirsttable}
\end{table}
and we can also look at confusion matrices generated in \autoref{fig:confusion_matrix}
\begin{figure}[h]
\centering
\includegraphics[scale=.5]{exp/all_phones_exp/svm_all_pairs_reg_plus_leehon_confusion_rcm.png}
\caption{Confusion Matrix for Classifications}
\label{fig:confusion_matrix}
\end{figure}
and in that plot we capped the diagonal entries (correct classifications) at 200
in order to better visualize the mistakes since in this plot the
correct classifications would normally dominate the plot and the misakes would be very faint.  Each axis has all 39 phones used in the origin Lee and Hon experiments (based on the TIMIT labels).  Entry $i,j$ indicates the number of times phone $i$ was classified as phone $j$.  Thus, the matrix is not symmetric.

We then tested a standard multiclass SVM on the same data and
achieved an accuracy of only \input{exp/liblinear_all_phones/krammer_multiclass_out}.

\section{Likelihood Ratio Test on Edges}

The error rate achieved on the development data using the likelihood
ratio test was \input{exp/bernoulli_all_phones/bernoulli_dev_leehon_errorrate.txt} which is considerably worse than the performance by
the SVM shown in the previous section.  The confusion matrix is
visualized in \autoref{fig:bernoulli_confusion_matrix}
\begin{figure}[h]
\centering
\includegraphics[scale=.5]{exp/bernoulli_all_phones/bernoulli_dev_leehon_confusion_rcm.png}
\caption{Bernoulli Confusion Matrix for Classifications}
\label{fig:bernoulli_confusion_matrix}
\end{figure}


We also ran a number of experiments investigating whether the SVM could be incorporated to improve the likelihood ratio test results, 
but no combination surpassed the resuls obtained with the
SVM result.

\section{Comparison of Different Methods}

\begin{table}[h]
  \centering
  \begin{tabular}{| c | c | c |}
    \hline
     Classifier & Features & Accuracy \\ \hline\hline
     Weighted kNN & Wavelets (Anden and Mallat) & \input{exp/scat_msc_knn/valid_top15_weighted_deverror.txt} \\
     \hline
     $1v1$ SVM & Edges on Spectrogram  & \input{exp/all_phones_exp/svm_little_reg_leehon_errorrate.txt} \\
     \hline
     2-Mixture Multiclass SVM \ref{sec:mixture-svm-pegasos} & Edges on Spectrogram  & \input{exp/pegasos/leehon39_warm_train_pegasos_dev_10Ninit_2C_0.037037lambda_Prj_error_rate.txt}\\
     \hline
     $1vRest$ SVM & Wavelets (Anden and Mallat) & \input{exp/scat_msc_exp/dev_accuracy_1_0.0001_-9.75} \\ 
     \hline
     $1v1$ Pegasos SVM (warm-start) & Edges on Spectrogram & \input{exp/pegasos/leehon_confusion_matrix_dev_output_1C_Q.txt}\\
     \hline
     Mixtures & Parts on Spectrogram & \input{exp/bernoulli_all_phones_parts/bernoulli_parts_error_rate_2r_70p.txt} \\
     \hline
     6 Mixtures & Edges on Spectrogram & \input{exp/bernoulli_all_phones/bernoulli_dev_leehon_errorrate.txt} \\
     \hline
     40 Mixtures, 10 Inits & Edges on Spectrogram & \input{exp/pegasos/leehon39_bmm_dev_10Ninit_40C_error_rate.txt} \\
     \hline
     $1vRest$ SVM & HOG & \input{exp/scat_msc_exp_hog/dev_accuracy_2_0.0001_-3} \\
     \hline
     $1vRest$ SVM & Spectrogram & \input{exp/spectrogram_exp/dev_accuracy_1_0.0001_-6} \\
     \hline
     $1vRest$ SVM & Multiscale Edges & \input{/home/mark/Research/phoneclassification/exp/multiscale_edges_exp/error} \\
     \hline
     Mixtures & Edges on Wavelets & \input{/home/mark/Research/phoneclassification/exp/scat_msc_exp_edges/error_rate} \\
     \hline

  \end{tabular}
  \caption{Error rates for different classifiers and features}
  \label{tab:comparison}
\end{table}

These comparisons of results, are, however, very sensitive to the
quantities of examples.  In particular the most common mistake is
\texttt{ix} versus \texttt{iy} but that is, in part, because
they are among the most common phones in the dataset.  In gauging
the predictive usefulness of a model on a particular language
it makes sense to focus ones efforts on preventing mistakes among
the most common syllables, but if we are going to understand
the behavior of these different methods on different classes of
phones then we want to understand how they compare on a per-class
basis.

\section{Results for different numbers of components in the Bernoulli mixture model}

We compare the effect of components on results and also compare what happens
when we use different numbers of random initializations (i.e. we run the algorithm
multiple times with different random starts). The results
can be seen in \autoref{tab:bmm-initializations-nmix}.
\begin{table}[h]
  \centering
  \begin{tabular}{| c | c |  c |}
    \hline
     \# components & \# Initializations  & Error Rate \\ \hline\hline
     2 & 5 & \input{exp/pairwise_bernoulli_thresh/leehon39_bmm_dev_5Ninit_2C_error_rate.txt} \\
     \hline
     3 & 5 & \input{exp/pairwise_bernoulli_thresh/leehon39_bmm_dev_5Ninit_3C_error_rate.txt} \\
     \hline
     6 & 5 & \input{exp/pairwise_bernoulli_thresh/leehon39_bmm_dev_5Ninit_6C_error_rate.txt} \\
     \hline
     9 & 5 &  \input{exp/pairwise_bernoulli_thresh/leehon39_bmm_dev_5Ninit_9C_error_rate.txt} \\
     12 & 5 &  \input{exp/pairwise_bernoulli_thresh/leehon39_bmm_dev_5Ninit_12C_error_rate.txt} \\
     \hline
     2 & 10 & \input{exp/pegasos/leehon39_bmm_dev_10Ninit_2C_error_rate.txt} \\
     \hline
     3 & 10 & \input{exp/pegasos/leehon39_bmm_dev_10Ninit_3C_error_rate.txt} \\
     \hline
     6 & 10 & \input{exp/pegasos/leehon39_bmm_dev_10Ninit_6C_error_rate.txt} \\
     \hline
     9 & 10 &  \input{exp/pegasos/leehon39_bmm_dev_10Ninit_9C_error_rate.txt} \\
     12 & 10 &  \input{exp/pegasos/leehon39_bmm_dev_10Ninit_12C_error_rate.txt} \\
     \hline
     15 & 10 & \input{exp/pegasos/leehon39_bmm_dev_10Ninit_15C_error_rate.txt} \\
     \hline
     18 & 10 &  \input{exp/pegasos/leehon39_bmm_dev_10Ninit_18C_error_rate.txt} \\
     21 & 10 &  \input{exp/pegasos/leehon39_bmm_dev_10Ninit_21C_error_rate.txt} \\
     \hline
     24 & 10 &  \input{exp/pegasos/leehon39_bmm_dev_10Ninit_24C_error_rate.txt} \\
     \hline
     27 & 10 &  \input{exp/pegasos/leehon39_bmm_dev_10Ninit_27C_error_rate.txt} \\
     \hline
     40 & 10 &  \input{exp/pegasos/leehon39_bmm_dev_10Ninit_40C_error_rate.txt} \\
     \hline
     50 & 10 &  \input{exp/pegasos/leehon39_bmm_dev_10Ninit_50C_error_rate.txt} \\
     \hline
     60 & 10 &  \input{exp/pegasos/leehon39_bmm_dev_10Ninit_60C_error_rate.txt} \\
     \hline
     70 & 10 &  \input{exp/pegasos/leehon39_bmm_dev_10Ninit_70C_error_rate.txt} \\
     \hline
  \end{tabular}
  \caption{Bernoulli Mixture Error Rates with different numbers of initializations}
  \label{tab:bmm-initializations-nmix}
\end{table}
The variance makes us wonder whether the recent \texttt{k-means++} algorithm
might be able to improve results.

\section{Overfitting the Training Data}

\section{Spreading differently across different frequency bands}

\section{Weight Thresholding and 1vs1}

We seek to develop a technique to threshold the entries to create a 1 versus 1 generative technique. We have noticed
that in 1 versus 1 classification the SVM performs much better than in the 1 versus rest. We want to develop
a technique that will perform 1 versus 1 using generative models.  Additionally, we want to have classifiers that 
are especially attuned to the minute differences between two models and focus only on those areas that
are ambiguous between two phone classes.

We consider a model where we estimate a single component per phone class.  For every pair of phone classes
we compute the log-likelihood ratio filter and threshold all of the small entries.  This provides a unique-pairwise
classifier.

For testing we run every pairwise classifier on an example and whichever class gets the most votes is the predicted class.

Our setting for this problem is that we have binary data paired
with a class
$\{(y_n,X_n)\}_{n=1}^N$ so that $y_n\in \{1,\ldots,C\}$ and
$X_n\in \{0,1\}^D$.  For each class $c$ with $1\leq c\leq C$
we have $N_c<  N$ data $\{(y_n,X_n)\}_{n\in\mathcal{N}_c}$
where $y_n=c$ for all $n\in\mathcal{N}_c$.  In the Naive Bayes
model we estimate a class-dependent probability distribution
for the data where
\begin{equation}
\mathbb{P}_c(X = x) = \prod_{d=1}^D \theta_c^{x(d)}(1-\theta_c(d)^{1-x(d)}
\end{equation}
with our estimate being 
\begin{equation}
\hat{\theta}_c(d) = \frac{1}{N_c}\sum_{n\in\mathcal{N}_c} X_n(d).
\end{equation}
Under that model the log-likelihood ratio test between whether
$X$ comes from class $c$ or $c'$ is
\begin{equation}\label{eq:linearloglike}
\log\frac{\mathbb{P}_c(X=x)}{\mathbb{P}_{c'}(X=x)} =
\sum_{d=1}^D x(d)\log\frac{\theta_c(d)(1-\theta_{c'}(d))}{\theta_{c'}(d)(1-\theta_c(d))} + \log\frac{1-\theta_c(d)}{1-\theta_{c'}(d)}
\end{equation}
and we may write \autoref{eq:linearloglike} as a linear classifier
\begin{equation}
\log\frac{\mathbb{P}_c(X=x)}{\mathbb{P}_{c'}(X=x)} = w^\top x+b
\end{equation}
where
\begin{align*}
w(d) &= \log\frac{\theta_c(d)(1-\theta_{c'}(d))}{\theta_{c'}(d)(1-\theta_c(d))}\\
b &= \sum_{d=1}^D \log\frac{1-\theta_c(d)}{1-\theta_{c'}(d)}
\end{align*}
for $1\leq d\leq D$.  Our estimate of the log-likelihood ratio
linear classifier $w$ that discriminates between class $c$ and
$c'$ is written $\hat{w}_{c,c'}$.  A thresholding procedure
consists in a binary vector-valued function $\mathcal{A}(N_c,\theta_c,N_{c'},\theta_{c'})\in \{0,1\}^D$ such that
\begin{equation}
\hat{w}_{c,c'}(d) = \begin{cases}
\log\frac{\theta_c(d)(1-\theta_{c'}(d))}{\theta_{c'}(d)(1-\theta_c(d))} & \mathcal{A}(N_c,\theta_c,N_{c'},\theta_{c'})(d) =1\\
0 & \text{otherwise}
\end{cases}
\end{equation}
with a similar procedure for $\hat{b}_{c,c'}$ based on the 
same criterion.
Note that in the case where
\begin{equation}
\theta_c(d)=\theta_{c'}(d) \Leftrightarrow \frac{\theta_c(d)(1-\theta_{c'}(d))}{\theta_{c'}(d)(1-\theta_c(d))} = \frac{1-\theta_c(d)}{1-\theta_{c'}(d)} = 1 
\end{equation}
we have $w_{c,c'}=0$ so $\mathcal{A}(N_c,\theta_c,N_{c'},\theta_{c'})$ may be viewed as a hypothesis test for the difference of means in
two binomial populations
\begin{align*}
H_0(c,c',d)\colon& \theta_c(d)=\theta_{c'}(d)\\
H_1(c,c',d)\colon& \theta_c(d)\neq \theta_{c'}(d)
\end{align*}
since 
$N_c\hat{\theta}_c(d)\sim\newoperatorname{Binomial}(N_c,\theta_c)$
for each class $c$.
Since we are dealing with large populations we may use a normal
approximation to the binomial and construct a test using
a two-sided Z-test procedure with pooled sample proportion
\begin{equation}
\hat{\theta}_{c,c'}(d) = \frac{N_c\hat{\theta}_c(d) +N_{c'}\hat{\theta}_{c'}(d) }{N_c+N_{c'}}
\end{equation}
and standard error
\begin{equation}
\hat{SE}_{c,c'}(d) = \sqrt{ \hat{\theta}_{c,c'}(d) (1-\hat{\theta}_{c,c'}(d))
\frac{N_c+N_{c'}}{N_cN_{c'}}}
\end{equation}
so that the test statistic
\begin{equation}
\hat{z}_{c,c'}(d) = \frac{\hat{\theta}_c(d) - \hat{\theta}_{c'}(d)}{\hat{SE}_{c,c'}(d)}
\end{equation}
has approximately $N(0,1)$ distribution. Simple thresholding for
a single coordinate $\hat{w}_{c,c'}(d)$ would pick a significance
level $\alpha$ and have
\begin{equation}
\hat{w}_{c,c'}(d) = \begin{cases}
\log\frac{\hat{\theta}_c(d)(1-\hat{\theta}_{c'}(d))}{\hat{\theta}_{c'}(d)(1-\hat{\theta}_c(d))} & 2\Phi(\hat{z}_{c,c'}(d))< \alpha\\
0 & \text{otherwise}
\end{cases}
\end{equation}
where $\Phi$ is the cumulative distribution function for the 
standard normal distribution. However, we are in a multiple
testing scenario where we are performing such a hypothesis test
for every pair of classes $c,c'$ and coordinate $d$.  A simple
Bonferroni correction procedure would mean that we would use
an adjusted $\alpha$ of the form
\begin{equation}
\frac{\alpha}{{C \choose 2} D}
\end{equation}
which would mean that we would fail to reject the null
hypothesis for essentially every test.



\subsection{Quantiles}

%
% TODO: - put plots into a single float
%       - write up what the actual chosen quantiles that I thresholded at were
%

We could also individually threshold on the basis of quantiles
this serves as a baseline to understand whether the False-discovery
rate or other techniques add anything.  With the quantile
thresholding procedure we pick a quantile $\tau$ and for every
pair of hypotheses we threshold everything below that quantile.
We start with $\tau \in \{0.001, 0.01,0.05,0.10,0.20,0.30,0.40\}$
and see the results obtained for these different values 
of $\tau$.

We find that we generally hurt performance by working with quantiles
\begin{table}[h]
  \centering
  \begin{tabular}{| l |  r |}
    \hline
     Quantile & Error Rate \\ \hline\hline
0 & \input{exp/pairwise_bernoulli_thresh/leehon_confusion_matrix_dev_output_1C_0Q.txt} \\ \hline
0.0000001 & \input{exp/pairwise_bernoulli_thresh/leehon_confusion_matrix_dev_output_1C_0.0000001Q.txt} \\ \hline
0.000001 & \input{exp/pairwise_bernoulli_thresh/leehon_confusion_matrix_dev_output_1C_0.000001Q.txt} \\ \hline
0.00001 & \input{exp/pairwise_bernoulli_thresh/leehon_confusion_matrix_dev_output_1C_0.00001Q.txt} \\ \hline
0.0001 & \input{exp/pairwise_bernoulli_thresh/leehon_confusion_matrix_dev_output_1C_0.0001Q.txt} \\ \hline

0.001 & \input{exp/pairwise_bernoulli_thresh/leehon_confusion_matrix_dev_output_1C_0.001Q.txt} \\ \hline
0.01 & \input{exp/pairwise_bernoulli_thresh/leehon_confusion_matrix_dev_output_1C_0.01Q.txt} \\ \hline
0.05 & \input{exp/pairwise_bernoulli_thresh/leehon_confusion_matrix_dev_output_1C_0.05Q.txt} \\ \hline
0.1 & \input{exp/pairwise_bernoulli_thresh/leehon_confusion_matrix_dev_output_1C_0.1Q.txt} \\ \hline
0.15 & \input{exp/pairwise_bernoulli_thresh/leehon_confusion_matrix_dev_output_1C_0.15Q.txt} \\ \hline
0.2 & \input{exp/pairwise_bernoulli_thresh/leehon_confusion_matrix_dev_output_1C_0.2Q.txt} \\ \hline
0.25 & \input{exp/pairwise_bernoulli_thresh/leehon_confusion_matrix_dev_output_1C_0.25Q.txt} \\ \hline
0.3 & \input{exp/pairwise_bernoulli_thresh/leehon_confusion_matrix_dev_output_1C_0.3Q.txt} \\ \hline
0.4 & \input{exp/pairwise_bernoulli_thresh/leehon_confusion_matrix_dev_output_1C_0.4Q.txt} \\ \hline
  \end{tabular}
  \caption{Averaged Pairwise SVM Error rates for different threshold quantiles over TIMIT development set}
  \label{tab:pairwiseSVM_simple_quantile_results}
\end{table}
as seen in \autoref{tab:pairwiseSVM_simple_quantile_results} when comparing results overall.  

\subsubsection{Pairwise Results}
\label{sec:pairwise-results}
In order to get a better sense
of the performance we also want to look at how performance of the
classifier changes on just the two class problem.  In the current
setting we require that all the classifiers vote on which phone,
but another case is where we know the pair and we are interested
in how the thresholded classifier performs on the pairwise classification task.

The output from this task is a pairwise loss matrix $M_q$ for each
quantile $q$ where we have
a row and column for each phone class so that for each pair
of classes $c\neq c'$ we have $M_q(c,c')$ is the number of times
data from class $c$ is classified as $c'$ by the classifier $w_{c,c'}$.  Generally speaking classification was good one question we have
is about the differences  $M_q(c,c') - M_{q'}(c,c')$ particularly
if we set the quantile $q'=0$ so that $M_{q'}(c,c')$ are the
pairwise results without thresholding.  The comparison
$M_q - M_0$ shows how thresholding changes the pairwise results.
We are interested in the distribution of $M_q(c,c')-M_0(c,c')$
so we look at the value for every pair of classes $c,c'$ and construct
a KDE estimate.  These are plotted in
\autoref{fig:kdediff_pairwise_0.0000001}
\autoref{fig:kdediff_pairwise_0.000001}
\autoref{fig:kdediff_pairwise_0.00001}
\autoref{fig:kdediff_pairwise_0.0001}
\autoref{fig:kdediff_pairwise_0.001}
\autoref{fig:kdediff_pairwise_0.01}
\autoref{fig:kdediff_pairwise_0.05}
\autoref{fig:kdediff_pairwise_0.1}
\autoref{fig:kdediff_pairwise_0.15}
\autoref{fig:kdediff_pairwise_0.2}
\autoref{fig:kdediff_pairwise_0.25}
\autoref{fig:kdediff_pairwise_0.3}
\autoref{fig:kdediff_pairwise_0.4}
if thresholding worked very well then the distribution would be
centered on negative number since that would mean that $M_q$
made fewer mistakes than $M_0$ so $M_q(c,c')-M_0(c,c')<0$ for most
pairs of classes $c,c'$.  On the other hand a more positive skew
indicates that $M_q$ has more mistakes than $M_0$.  In general, we
see that as the thresholding increases there tends to be a positive
skew to the number of mistakes, so thresholding generally tends to
decrease performance.

\begin{figure}
\centering
\begin{subfigure}[h]{.3\textwidth}
    \centering
    \includegraphics[width=\textwidth]{{exp/pairwise_bernoulli_thresh/known_pair_confusion_matrix_compare_1C_0Q_0.0000001Q}.png}
    \caption{KDE over pairwise score differences for quantiles q=0.0000001 versus q=0}
    \label{fig:kdediff_pairwise_0.0000001}
    \end{subfigure} 
~
\begin{subfigure}[h]{.3\textwidth}
    \centering
    \includegraphics[width=\textwidth]{{exp/pairwise_bernoulli_thresh/known_pair_confusion_matrix_compare_1C_0Q_0.000001Q}.png}
    \caption{KDE over pairwise score differences for quantiles q=0.000001 versus q=0}
    \label{fig:kdediff_pairwise_0.000001}
    \end{subfigure}
~
\begin{subfigure}[h]{.3\textwidth}
    \centering
    \includegraphics[width=\textwidth]{{exp/pairwise_bernoulli_thresh/known_pair_confusion_matrix_compare_1C_0Q_0.00001Q}.png}
    \caption{KDE over pairwise score differences for quantiles q=0.00001 versus q=0}
    \label{fig:kdediff_pairwise_0.00001}
    \end{subfigure}
~\begin{subfigure}[h]{.3\textwidth}
    \centering
    \includegraphics[width=\textwidth]{{exp/pairwise_bernoulli_thresh/known_pair_confusion_matrix_compare_1C_0Q_0.0001Q}.png}
    \caption{KDE over pairwise score differences for quantiles q=0.0001 versus q=0}
    \label{fig:kdediff_pairwise_0.0001}
    \end{subfigure}
~\begin{subfigure}[h]{.3\textwidth}
    \centering
    \includegraphics[width=\textwidth]{{exp/pairwise_bernoulli_thresh/known_pair_confusion_matrix_compare_1C_0Q_0.001Q}.png}
    \caption{KDE over pairwise score differences for quantiles q=0.001 versus q=0}
    \label{fig:kdediff_pairwise_0.001}
    \end{subfigure}
~\begin{subfigure}[h]{.3\textwidth}
    \centering
    \includegraphics[width=\textwidth]{{exp/pairwise_bernoulli_thresh/known_pair_confusion_matrix_compare_1C_0Q_0.01Q}.png}
    \caption{KDE over pairwise score differences for quantiles q=0.01 versus q=0}
    \label{fig:kdediff_pairwise_0.01}
    \end{subfigure}
\end{figure}
\begin{figure}
\begin{subfigure}{.3\textwidth}
    \centering
    \includegraphics[width=\textwidth]{{exp/pairwise_bernoulli_thresh/known_pair_confusion_matrix_compare_1C_0Q_0.05Q}.png}
    \caption{KDE over pairwise score differences for quantiles q=0.05 versus q=0}
    \label{fig:kdediff_pairwise_0.05}
    \end{subfigure}
~\begin{subfigure}{.3\textwidth}
    \centering
    \includegraphics[width=\textwidth]{{exp/pairwise_bernoulli_thresh/known_pair_confusion_matrix_compare_1C_0Q_0.1Q}.png}
    \caption{KDE over pairwise score differences for quantiles q=0.1 versus q=0}
    \label{fig:kdediff_pairwise_0.1}
    \end{subfigure}
~\begin{subfigure}{.3\textwidth}
    \centering
    \includegraphics[width=\textwidth]{{exp/pairwise_bernoulli_thresh/known_pair_confusion_matrix_compare_1C_0Q_0.15Q}.png}
    \caption{KDE over pairwise score differences for quantiles q=0.15 versus q=0}
    \label{fig:kdediff_pairwise_0.15}
    \end{subfigure}
~\begin{subfigure}{.3\textwidth}
    \centering
    \includegraphics[width=\textwidth]{{exp/pairwise_bernoulli_thresh/known_pair_confusion_matrix_compare_1C_0Q_0.2Q}.png}
    \caption{KDE over pairwise score differences for quantiles q=0.2 versus q=0}
    \label{fig:kdediff_pairwise_0.2}
    \end{subfigure}
~\begin{subfigure}{.3\textwidth}
    \centering
    \includegraphics[width=\textwidth]{{exp/pairwise_bernoulli_thresh/known_pair_confusion_matrix_compare_1C_0Q_0.25Q}.png}
    \caption{KDE over pairwise score differences for quantiles q=0.25 versus q=0}
    \label{fig:kdediff_pairwise_0.25}
    \end{subfigure}
~\begin{subfigure}{.3\textwidth}
    \centering
    \includegraphics[width=\textwidth]{{exp/pairwise_bernoulli_thresh/known_pair_confusion_matrix_compare_1C_0Q_0.3Q}.png}
    \caption{KDE over pairwise score differences for quantiles q=0.3 versus q=0}
    \label{fig:kdediff_pairwise_0.3}
    \end{subfigure}
~\begin{subfigure}{.3\textwidth}
    \centering
    \includegraphics[width=\textwidth]{{exp/pairwise_bernoulli_thresh/known_pair_confusion_matrix_compare_1C_0Q_0.4Q}.png}
    \caption{KDE over pairwise score differences for quantiles q=0.4 versus q=0}
    \label{fig:kdediff_pairwise_0.4}
    \end{subfigure}
\end{figure}
\subsection{FDR}

One way arround the overly-conservative error rates is to use
the Benjamini-Hochberg procedure describe in 
\cite{bhoch95} where the threshold is set so that the 
expected false discovery rate is below some threshold. The
false discovery rate is defined as $F/R$ where
$F$ is the number of falsely rejected null hypotheses and
$R$ is the number of rejected null hypotheses.  In this procedure
we collect a set of p-values $P_{c,c',d}=2\Phi(\hat{z}_{c,c'}(d))$
for every pair of classes $c$, $c'$ and coordinate $d$.  We then
compute the order statistics $P_{(1)} <P_{(2)} <\cdots <P_{(K)}$
of the p-values and we let $k_*$ be the largest $k$ such that
\begin{equation}
P_{(k)} \leq \frac{k}{K} \alpha
\end{equation}
where $\alpha$ is the desired expected false discovery rate upperbound.  We then reject every hypothesis $P_{(k)}$ for $1\leq k\leq k_*$.
Using this procedure we find that there is a small increase
in performance associated with using a very conservative $\alpha$
threshold as seen in \autoref{tab:pairwise_bhfdr_thresholding_results}
\begin{table}
  \centering
  \begin{tabular}{| l |  r |}
    \hline
     Quantile & Error Rate \\ \hline\hline
0.0000001 & \input{exp/pairwise_bernoulli_thresh/leehon_confusion_matrix_dev_output_1C_0.0000001Q_bhfdr.txt} \\ \hline
0.000001 & \input{exp/pairwise_bernoulli_thresh/leehon_confusion_matrix_dev_output_1C_0.000001Q_bhfdr.txt} \\ \hline
0.00001 & \input{exp/pairwise_bernoulli_thresh/leehon_confusion_matrix_dev_output_1C_0.00001Q_bhfdr.txt} \\ \hline
0.0001 & \input{exp/pairwise_bernoulli_thresh/leehon_confusion_matrix_dev_output_1C_0.0001Q_bhfdr.txt} \\ \hline

0.001 & \input{exp/pairwise_bernoulli_thresh/leehon_confusion_matrix_dev_output_1C_0.001Q_bhfdr.txt} \\ \hline
0.01 & \input{exp/pairwise_bernoulli_thresh/leehon_confusion_matrix_dev_output_1C_0.01Q_bhfdr.txt} \\ \hline
0.05 & \input{exp/pairwise_bernoulli_thresh/leehon_confusion_matrix_dev_output_1C_0.05Q_bhfdr.txt} \\ \hline
0.1 & \input{exp/pairwise_bernoulli_thresh/leehon_confusion_matrix_dev_output_1C_0.1Q_bhfdr.txt} \\ \hline
0.15 & \input{exp/pairwise_bernoulli_thresh/leehon_confusion_matrix_dev_output_1C_0.15Q_bhfdr.txt} \\ \hline
0.2 & \input{exp/pairwise_bernoulli_thresh/leehon_confusion_matrix_dev_output_1C_0.2Q_bhfdr.txt} \\ \hline
0.25 & \input{exp/pairwise_bernoulli_thresh/leehon_confusion_matrix_dev_output_1C_0.25Q_bhfdr.txt} \\ \hline
0.3 & \input{exp/pairwise_bernoulli_thresh/leehon_confusion_matrix_dev_output_1C_0.3Q_bhfdr.txt} \\ \hline
0.4 & \input{exp/pairwise_bernoulli_thresh/leehon_confusion_matrix_dev_output_1C_0.4Q_bhfdr.txt} \\ \hline
  \end{tabular}
  \caption{Averaged Pairwise SVM Error rates for different threshold quantiles using the BH FDR thresholding technique over TIMIT development set}
  \label{tab:pairwise_bhfdr_thresholding_results}
\end{table}

We can analyze the pairwise results in a similar manner to \autoref{sec:pairwise-results} and we find that there is a slight negative skew to the results although graphically it appears that the BH FDR procedure increases pairwise mistakes about as often as it decreases
pairwise mistakes.  The plots are in
\autoref{fig:kdediff_pairwise_bhfdr_0.0000001}
\autoref{fig:kdediff_pairwise_bhfdr_0.000001}
\autoref{fig:kdediff_pairwise_bhfdr_0.00001}
\autoref{fig:kdediff_pairwise_bhfdr_0.0001}
\autoref{fig:kdediff_pairwise_bhfdr_0.001}
\autoref{fig:kdediff_pairwise_bhfdr_0.01}
\autoref{fig:kdediff_pairwise_bhfdr_0.05}
\autoref{fig:kdediff_pairwise_bhfdr_0.1}
\autoref{fig:kdediff_pairwise_bhfdr_0.15}
\autoref{fig:kdediff_pairwise_bhfdr_0.2}
\autoref{fig:kdediff_pairwise_bhfdr_0.25}
\autoref{fig:kdediff_pairwise_bhfdr_0.3}
\autoref{fig:kdediff_pairwise_bhfdr_0.4}
Recall that
if thresholding worked very well then the distribution would be
centered on negative number since that would mean that $M_q$
made fewer mistakes than $M_0$ so $M_q(c,c')-M_0(c,c')<0$ for most
pairs of classes $c,c'$.  On the other hand a more positive skew
indicates that $M_q$ has more mistakes than $M_0$.  In general, we
see that as the thresholding increases there tends to be a positive
skew to the number of mistakes, so thresholding generally tends to
decrease performance.

\begin{figure}
\centering
\begin{subfigure}{.25\textwidth}
    \centering
    \includegraphics[width=\textwidth]{{exp/pairwise_bernoulli_thresh/known_pair_confusion_matrix_compare_1C_0Q_simple_0.0000001Q_bhfdr}.png}
    \caption{KDE over pairwise score differences for BH FDR thresholding at q=0.0000001 versus q=0}
    \label{fig:kdediff_pairwise_bhfdr_0.0000001}
    \end{subfigure}
~\begin{subfigure}{.25\textwidth}
    \centering
    \includegraphics[width=\textwidth]{{exp/pairwise_bernoulli_thresh/known_pair_confusion_matrix_compare_1C_0Q_simple_0.000001Q_bhfdr}.png}
    \caption{KDE over pairwise score differences for BH FDR thresholding at q=0.000001 versus q=0}
    \label{fig:kdediff_pairwise_bhfdr_0.000001}
    \end{subfigure}
~\begin{subfigure}{.25\textwidth}
    \centering
    \includegraphics[width=\textwidth]{{exp/pairwise_bernoulli_thresh/known_pair_confusion_matrix_compare_1C_0Q_simple_0.00001Q_bhfdr}.png}
    \caption{KDE over pairwise score differences for BH FDR thresholding at q=0.00001 versus q=0}
    \label{fig:kdediff_pairwise_bhfdr_0.00001}
    \end{subfigure}
~\begin{subfigure}{.25\textwidth}
    \centering
    \includegraphics[width=\textwidth]{{exp/pairwise_bernoulli_thresh/known_pair_confusion_matrix_compare_1C_0Q_simple_0.0001Q_bhfdr}.png}
    \caption{KDE over pairwise score differences for BH FDR thresholding at q=0.0001 versus q=0}
    \label{fig:kdediff_pairwise_bhfdr_0.0001}
    \end{subfigure}
~\begin{subfigure}{.25\textwidth}
    \centering
    \includegraphics[width=\textwidth]{{exp/pairwise_bernoulli_thresh/known_pair_confusion_matrix_compare_1C_0Q_simple_0.001Q_bhfdr}.png}
    \caption{KDE over pairwise score differences for BH FDR thresholding at q=0.001 versus q=0}
    \label{fig:kdediff_pairwise_bhfdr_0.001}
    \end{subfigure}
~\begin{subfigure}{.25\textwidth}
    \centering
    \includegraphics[width=\textwidth]{{exp/pairwise_bernoulli_thresh/known_pair_confusion_matrix_compare_1C_0Q_simple_0.01Q_bhfdr}.png}
    \caption{KDE over pairwise score differences for BH FDR thresholding at q=0.01 versus q=0}
    \label{fig:kdediff_pairwise_bhfdr_0.01}
    \end{subfigure}
\end{figure}


\begin{figure}
\begin{subfigure}{.27\textwidth}
  \centering
    \includegraphics[width=\textwidth]{{exp/pairwise_bernoulli_thresh/known_pair_confusion_matrix_compare_1C_0Q_simple_0.05Q_bhfdr}.png}
    \caption{KDE over pairwise score differences for BH FDR thresholding at q=0.05 versus q=0}
    \label{fig:kdediff_pairwise_bhfdr_0.05}
\end{subfigure}
~\begin{subfigure}{.27\textwidth}
    \centering
    \includegraphics[width=\textwidth]{{exp/pairwise_bernoulli_thresh/known_pair_confusion_matrix_compare_1C_0Q_simple_0.1Q_bhfdr}.png}
    \caption{KDE over pairwise score differences for BH FDR thresholding at q=0.1 versus q=0}
    \label{fig:kdediff_pairwise_bhfdr_0.1}
    \end{subfigure}
~\begin{subfigure}{.27\textwidth}
    \centering
    \includegraphics[width=\textwidth]{{exp/pairwise_bernoulli_thresh/known_pair_confusion_matrix_compare_1C_0Q_simple_0.15Q_bhfdr}.png}
    \caption{KDE over pairwise score differences for BH FDR thresholding at q=0.15 versus q=0}
    \label{fig:kdediff_pairwise_bhfdr_0.15}
    \end{subfigure}
~\begin{subfigure}{.27\textwidth}
    \centering
    \includegraphics[width=\textwidth]{{exp/pairwise_bernoulli_thresh/known_pair_confusion_matrix_compare_1C_0Q_simple_0.2Q_bhfdr}.png}
    \caption{KDE over pairwise score differences for BH FDR thresholding at q=0.2 versus q=0}
    \label{fig:kdediff_pairwise_bhfdr_0.2}
    \end{subfigure}
~\begin{subfigure}{.27\textwidth}
    \centering
    \includegraphics[width=\textwidth]{{exp/pairwise_bernoulli_thresh/known_pair_confusion_matrix_compare_1C_0Q_simple_0.25Q_bhfdr}.png}
    \caption{KDE over pairwise score differences for BH FDR thresholding at q=0.25 versus q=0}
    \label{fig:kdediff_pairwise_bhfdr_0.25}
    \end{subfigure}
~\begin{subfigure}{.27\textwidth}
    \centering
    \includegraphics[width=\textwidth]{{exp/pairwise_bernoulli_thresh/known_pair_confusion_matrix_compare_1C_0Q_simple_0.3Q_bhfdr}.png}
    \caption{KDE over pairwise score differences for BH FDR thresholding at q=0.3 versus q=0}
    \label{fig:kdediff_pairwise_bhfdr_0.3}
    \end{subfigure}
~\begin{subfigure}{.27\textwidth}
    \centering
    \includegraphics[width=\textwidth]{{exp/pairwise_bernoulli_thresh/known_pair_confusion_matrix_compare_1C_0Q_simple_0.4Q_bhfdr}.png}
    \caption{KDE over pairwise score differences for BH FDR thresholding at q=0.4 versus q=0}
    \label{fig:kdediff_pairwise_bhfdr_0.4}
\end{subfigure}
\end{figure}



\subsection{Worst Case Loss}

The approach to handle the worst case loss is as follows.
We only want to incorporate coordinates that we predict are 
useful for making decisions.  A coordinate $d$ is not useful
for discriminating between models $c$ and $c'$ if there is
a mixture of bernoulli random variables with proportions
$\eta_c,\eta_{c'}$ and respective mixture weights $\pi,1-\pi$
whose likelihood for the observed data in coordinate
$d$ is reasonably large but for whom the expected loss exceeds
some threshold.

Formally, we construct a confidence region for the proportions
and the mixture weights which is a joint confidence region
for all three parameters.  Each of the confidence intervals
are simply binomial proportion. There are a number of different
confidence intervals to use, for example we could
use the likelihood ratio interval inverted interval
we will not use the Wald interval for reasons discussed in 
Tony Cai's paper

\subsection{Multiple Models}

The case of mixture models is somewhat more complicated since
we need to use weighted p-value tests since we don't know
to which mixture component each data point belongs and
we have to use that uncertainty.  Before we do the thresholding
we first estimate several models $K_c$ for each class $c$.
The linear classifiers are constructed between all pairs from
different classes, i.e. we have a classifier $(\hat{w}_{c,k,c',k'},\hat{b}_{c,k,c',k'})$
for every $1\leq k\leq K_c$, $1\leq k'\leq K_{c'}$ and every pair
$c$,$c'$ of classes. The total number of classifiers is
\begin{equation}
\sum_{c=1}^{C-1} K_c \sum_{c'=c+1}^C K_{c'}
\end{equation}
which becomes very large for mixture models.



\section{ Multiclass Pegasos}

\subsection{Using Pegasos to finish estimation of Linear Classifiers}

Based on \autoref{tab:comparison} the most successful method
used an SVM to learn pairwise discrimination methods, however we
found that training took a very long time.  In this section
we consider an attempt to speed up training using a combination
of generative estimation and the Pegasos SVM solver \cite{pegasos}.
We are in the same setting with binary data coming from two class
$c$ and $c'$ with $1\leq c<c'\leq C$. We denote the data
from class $c$ as $X_c\in \{0,1\}^{N_c\times D}$ and
the data from class $c'$ as $X_{c'}\in \{0,1\}^{N_{c'}\times D}$.
We use simple averages to estimate naive Bayes models for
both $\hat{\theta}_c \in (0,1)^D$ and $\hat{\theta}_{c'}\in(0,1)^D$ where
\begin{equation}
\hat{\theta}_j(d) \begin{cases}
\frac{1}{N_j} \sum_{i=1}^{N_j} X(i,d) &  0 < \sum_{i=1}^{N_j} X(i,d) < N_j\\
\frac{1}{2N_j} & \sum_{i=1}^{N_j} X(i,d) = 0\\
1 - \frac{1}{2N_j} & \text{otherwise}
\end{cases}
\end{equation}
for each coordinate $1\leq d\leq D$ and $j=c,c'$. We warm start
our Pegasos solver with a linear classifier $(w,b)\in\mathbb{R}^D\times \mathbb{R}$ where
\begin{align}
w(d) =& \log \frac{\hat{\theta}_c(d)(1-\hat{\theta}_{c'}(d))}{\hat{\theta}_{c'}(d)(1-\hat{\theta}_{c}(d))}\\
b =& \sum_{d=1}^D \log \frac{1-\hat{\theta}_c(d)}{1-\hat{\theta}_{c'}(d)}
\end{align} 
for each coordinate $1\leq d\leq D$.  With this warm start
the solver converges quickly and we can estimate pairwise classifiers
for all 1182 pairs of classes rapidly.  Our results show a small improvement results from doing this with a result of \input{exp/pegasos/leehon_confusion_matrix_dev_output_1C_Q.txt}.  One question is
how the pegasos training changes the weight vectors and also
what the change in the pairwise classification rate looks like.

\subsection{Change in Pairwise Classification Rates}

We use a similar technique as \autoref{sec:pairwise-results} to
get the change in pairwise classification rates.  Namely, we constrct
a matrix $M^{pegasos}$ where $M^{pegasos}(c,c')$ is the number of times
the classifier $(w_{c,c'},b_{c,c'})$ (as estimated by Pegasos)
classifies training data from class $c$ as coming from class
$c'$ so it quantifies the non-oblivious pairwise loss. We note
this is very different than the oblivious loss we compute
in the standard classification procedure since we use all the pairwise
classifier to vote whereas in this case we are focusing on a single
pairwise classifier and just two classes.  The KDE plot shows a
significant negative skew in \autoref{fig:kdediff_pairwise_1C_pegasos_60T_1000k_2l} which shows how Pegasos is able to significantly
reduce the number of mistakes across many of the pairwise differences with some pairwise mistakes being reduced by a large amount.
\begin{figure}[ht]

    \centering
    \includegraphics[scale=.5]{{exp/pegasos/known_pair_confusion_matrix_compare_1C_0Q_simple_1C_pegasos_60T_1000k_2l}.png}
    \caption{KDE over pairwise mistake differences for Pegasos-trained warm-start classifier compared to pure warm-start classifier (naive Bayes for warm-start).  Notice the significant negative skew showing how Pegasos decreaseses the number of mistakes more than it increases them.}
    \label{fig:kdediff_pairwise_1C_pegasos_60T_1000k_2l}
\end{figure}


\subsection{Changes in the weight vector}

We now wish to analyze how Pegasos training modifies the weight
vectors initialized using a naive Bayes procedure.  The first question
to look at is whether there are any trends.  The first is whether
there is an overall difference between the means of the two

\section{Pegasos 1vRest with multiple components}

Another iteration of these is to do a warm start with multiple
components and then train one versus rest with preclustered data.
The setup is as follows we have data pairs
$\{(x_i,y_i)\}_{i=1}^N$ where $1\leq y_i\leq C$ for $1\leq i\leq N$.
We have $x_i\in\{0,1\}^D$ for $1\leq i\leq N$.  Let $I_c$ denote
the set of all indices $i$ such that $y_i = c$ then we can construct
a class-specific data matrix $X_c\in \{0,1\}^{N_c\times D}$ where
$N_c=|I_c|$.  The algorithm we use is a multi-class Pegasos
\cite{multipegasos} where we estimate a matrix $W$ of class-specific
weight vectors $(w_1,\ldots,w_c)^\top\in\mathbb{R}^{C\times D}$ so
that the classifier $f$ learned is
\begin{equation}
f(x) = \underset{1\leq c\leq C}{\arg\max} w_c^\top x.
\end{equation}
Following \cite{multipegasos} we use a multiclass hinge loss
$l\colon \mathbb{R}^{C\times D} \times \{0,1\}^D \times \{1,\ldots,C\}$
\begin{equation}\label{eq:multiclass_hinge_loss}
l(W,x,y) = \max\{ 0,1 + \max_{c\neq y} (w_c -w_y )^\top x\}
\end{equation}
where $y$ is the true underlying class of $x$. The multiclass
SVM objective then is
\begin{equation}
P(W,\{(x_i,y_i)\}_{i=1}^N) = \frac{\lambda}{2}\|W\|_F^2 +\frac{1}{N}\sum_{i=1}^N l(W,x_i,y_i)
\end{equation}
where $\|\cdot\|_F$ is the Froebenius matrix norm.  The multi-class
Pegasos works in rounds where there is first an update using the
subgradient followed by a projection step.  In each round
$t$ we minimize a stochastic objective
\begin{equation}
  P^{(t)}(W,\{(x_i,y_i)\}_{i\in I^{(t)}})=\frac{\lambda }{2}\|W\|^2_F +\frac{1}{K}\sum_{i\in I^{(t)}} l(W,x_i,y_i)
\end{equation}
where $I^{(t)}$ is a randomly chosen subset of size $K$ from $I_{y^{(t)}}$
where $y^{(t)}$ is a randomly chosen class (according to class 
frequency in the training data).  To minimize $P^{(t)}(W,\{x_i,y_i\}_{i\in I^{(t)}})$
we take a single projected subgradient step of the form
\begin{equation}\label{eq:pegasos_step}
W^{(t+1)} = \Pi_{B(\lambda)}(W^{(t)} -\eta^{(t)}\nabla^{(t)})
\end{equation}
where $\Pi_{B(\lambda)}$ projects the data onto the ball
\begin{equation}
B(\lambda)=\left\{ w\colon \|w\| \leq \frac{1}{\sqrt{\lambda}}\right\},
\end{equation}
$\eta^{(t)}=1/(\lambda t)$ is the learning rate, and $\nabla^{(t)}\in\mathbb{R}^{C\times D}$
is the sub-gradient matrix estimated from $P^{(t)}$. Due to the
piecewise nature of the hinge loss we get the following potential
subgradients for the row of $\nabla^{(t)}$ associated with class
$c$ is
\begin{equation}\label{eq:hinge_subgradient}
\nabla^{(t)}_c = \begin{cases}
\lambda W^{(t)}_c  & |I_c^{(t)}|=0\\
\lambda W^{(t)}_c - \frac{1}{|I_c^{(t)}|}\sum_{i\in I_c^{(t)}} x_i & c =y^{(t)}\\
\lambda W^{(t)}_c + \frac{1}{|I_c^{(t)}|}\sum_{i\in I_c^{(t)}} x_i & c \neq y^{(t)} 
\end{cases}
\end{equation}
where $I_c^{(t)}$ is the subset of $I^{(t)}$ of training examples that contribute to the loss:
\begin{equation}\label{eq:hinge_batch}
I_c^{(t)} = \begin{cases}
\left\{ x_i \in I^{(t)}\mid (w_c - w_{y^{(t)}})^\top x_i > -1,c=\underset{c'}{\arg\max}\; w_{c'}^\top x_i\right\} & c\neq y^{(t)}\\
\left\{ x_i \in I^{(t)}\mid \underset{c'}{\arg\max}\;(w_{c'} - w_{y^{(t)}})^\top x_i > -1\right\} & c = y^{(t)}.
\end{cases}
\end{equation}
Another possibility is to use the ramp loss which only uses those
examples that contribute to a small amount of loss
\begin{equation}\label{eq:ramp_batch}
I_{c,ramp}^{(t)} = \begin{cases}
\left\{ x_i \in I^{(t)}\mid 0 > (w_c - w_{y^{(t)}})^\top x_i > -1,c=\underset{c'}{\arg\max}\; w_{c'}^\top x_i\right\} & c\neq y^{(t)}\\
\left\{ x_i \in I^{(t)}\mid 0 > \underset{c'}{\arg\max}\;(w_{c'} - w_{y^{(t)}})^\top x_i > -1\right\} & c = y^{(t)}.
\end{cases}
\end{equation}
The projection $\Pi_{B(\lambda)}$ may then be performed as
\begin{equation}\label{eq:pegasos-multiclass-projection}
\Pi_{B(\lambda)}(W^{(t)}-\eta^{(t)}\nabla^{(t)}) = \min\left\{ 1,\frac{1/\sqrt{\lambda}}{\|W^{(t)}-\eta^{(t)}\nabla^{(t)}\|_F}\right\} (W^{(t)}-\eta^{(t)}\nabla^{(t)}).
\end{equation}
A summary of the algorithm in this case where we process a single example is given in
\autoref{alg:pegasos-single-multiclass}
\begin{algorithm}[t]
\SetAlgoNoLine
\KwIn{Data set $\mathcal{D}=\{(x,y)\}\subset \mathbb{R}^D\times [C]$, regularization parameter $\lambda > 0$, initial weight matrix $W\in\mathbb{R}^{C\times D}$, 
number of rounds $T$, loss function $l(W,x,y)$}
\KwOut{Weight matrix $W\in\mathbb{R}^{C\times D}$}
\For{$t=1,2,\ldots,T$}{
  Draw example $(x,y)$ uniformly at random from $\mathcal{D}$\;
  $\eta \leftarrow \frac{1}{\lambda t}$\;
  $W \leftarrow (1- \lambda\eta) W$\;
  \If{ $l(W,x,y) > 0$}{
    $c \leftarrow \arg\max_{c\neq y}(w_c - w_y)^\top x$\;
    \tcc{Adjust the rows of $W$ contributing to the loss}
    $w_c \leftarrow w_c - \eta x$\;
    $w_y \leftarrow w_y + \eta x$\;
   }
  \tcc{Optional Projection Step}
  $\left( W \leftarrow \min\left\{1,\frac{1/\sqrt{\lambda}}{\|W\|}\right\} W \right)\;

}

\caption{Multiclass Pegasos}
\label{alg:pegasos-single-multiclass}
\end{algorithm}




\subsection{Pegasos For Mixture SVM}
\label{sec:mixture-svm-pegasos}
A common variant of the SVM is to include  latent variables
as in the case of mixture of multivariate Bernoulli models.
The extension to latent models means that the weight matrix
$W$ has to accomodate weight vectors for the mixture components
so $W\in \mathbb{R}^{(K_1+K_2+\cdots+K_C)\times D}$ where $K_c$ is the
number of mixture components for class $c$.  We denote the
rows of $W$ by $w_{c,k}$ where $1\leq k\leq K_c$ so that they
are indexed by both a class and component. 

We have to adjust our
definition of the loss since 
\autoref{eq:multiclass_hinge_loss} involves a comparison of
the score for the true underlying class $y$ against the largest score
for another class $c$ and if the true class has $K_y >1$ components
then there is no unique score $w_y^\top x$ to compare against $w_c^\top x$. If we use the analogy to exponential-family models then
$w_{y,k}^\top x$ may be interpreted as the log probability (possibly unnormalized if we omit the constant term) of $x$ under mixture
$k$ for class $y$
and if we have mixture weights
$\pi_{y,k}$ for $1\leq k\leq K_y$ then the log-probability for class
$y$ is
\begin{equation}
s_y(x) = \log \sum_{k=1}^{K_y} \pi_{y,k}\exp( w_{y,k}^\top x)
\end{equation}
which is approximately equal to $\max_{k} w_{y,k}^\top x$. The
approximation is close as long
as the mixture components are sufficiently different and each of
the mixture weights is sufficiently close to uniform.
We define a mixture component-compatible version
of the multiclass hinge loss as
\begin{equation}\label{eq:mixture_multiclass_hinge_loss}
l(W,x,y) = \max\{0,1+\max_{c\neq y}\max_{1\leq k\leq K_c}\min_{1\leq k'\leq K_y}(w_{c,k}- w_{y,k'})^\top x\}
\end{equation}
where the $\min_{1\leq k'\leq K_y}$ ensure that we compare $w_{c,k}$
against the strongest weight vector $w_{y,k'}$ for the true class
$y$.  To minimize \autoref{eq:mixture_multiclass_hinge_loss}
we use mini-batch stochastic gradient descent steps of the
form \autoref{eq:pegasos_step}.  Optimization occurs in rounds
where in round $t$ we randomly choose a class $y^{(t)}$ and we select
a minibatch of examples $I^{(t)}$ from that class we also have an estimate
for the weight matrix $W^{(t)}$ from the previous round. 
The set $I^{(t)}$, in analogy to \autoref{eq:hinge_batch}, is further
partitioned into class-component specific subsets
$I^{(t)}_{c,k}$ where
\begin{equation}\label{eq:hinge_component_batch}
I_{c,k}^{(t)} = \begin{cases}
\left\{ x_i \in I^{(t)}\mid \min_{1\leq k'\leq K_{y^{(t)}}}(w_{c,k} - w_{y^{(t)},k})^\top x_i > -1,\;(c,k)=\underset{(c_0,k_0) \colon 1\leq k_0\leq K_{c_0}}{\arg\max}\; (w_{c_0,k_0})^\top x_i\right\} & c\neq y^{(t)}\\
\left\{ x_i \in I^{(t)}\mid \underset{\begin{subarray}{c}
(c',k')\colon c'\neq y^{(t)}\\
1\leq k'\leq K_{c'}\end{subarray}}{\arg\max}\;(w_{c',k'} - w_{y^{(t)},k})^\top x_i > -1,\; (w_{y^{(t)},k})^\top x_i = \max_{k''} (w_{y^{(t)},k''})^\top x_i\right\} & c = y^{(t)}.
\end{cases}
\end{equation}
The definition of $I_{c,k}^{(t)}$ may be broken into some components:
\begin{enumerate}
\item Only those points which contribute to the loss are included, 
so correctly classified examples away from the decision boundary
are not included.
\item If $c=y^{(t)}$ then $I_{c,k}^{(t)}$ contains those points $x_i\in I^{(t)}$ where
$(w_{c,k})^\top x$ is larger than 
$(w_{c,k'})^\top x$ for every other component $1\leq k'\leq K_y$.
\item If $c\neq y^{(t)}$ then $(w_{c,k})^\top x$ is larger than $(w_{c',k'})^\top x$ for every
other $(c',k')$ where $c'\neq y^{(t)}$.
\end{enumerate}
The above principles mean that we focus on the components and 
classes that produce the largest impact on the hinge loss.
We these sets to estimate the subgradient of
\begin{equation}
P(W,I^{(t)}) = \frac{\lambda}{2}\|W\|^2_F + \frac{1}{|I^{(t)}|}\sum_{i\in I^{(t)}} l(W,x_i,y_i)
\end{eqution}
taken at the currrent estimate of the weight matrix $W^{(t)}$.
Denote by $\nabla^{(t)}$ this subgradient which is defined
similarly as \autoref{eq:hinge_subgradient} but each
row $\nabla^{(t)}_{c,k}$ corresponds to a class-component pair $(c,k)$
(as are the rows of $W^{(t)}$) so that
\begin{equation}
\nabla^{(t)}_{c,k} = \begin{cases} \lambda W_{c,k}^{(t)} &  |I_{c,k}^{(t)}| = 0\\
\lambda W_{c,k}^{(t)} - \frac{1}{|I_{c,k}^{(t)}|} \sum_{i\in I_{c,k}^{(t)}} x_i &  c = y^{(t)}\\
\lambda W_{c,k}^{(t)} +  \frac{1}{|I_{c,k}^{(t)}|} \sum_{i\in I_{c,k}^{(t)}} x_i &  c\neq y^{(t)}.
\end{cases}
\end{equation}
The projection step of the algorithm is the same as \autoref{eq:pegasos-multiclass-projection}. We present a version of the algorithm
appropriate for handling a single example in \autoref{alg:pegasos-single-multiclass-multicomponent} where we assume that each class
$c$ has $K$ components (although it is simple to modify the version
where $K$ is allowed to vary across different classes)
\begin{algorithm}[t]
\SetAlgoNoLine
\KwIn{Data set $\mathcal{D}=\{(x,y)\}\subset \mathbb{R}^D\times [C]$, regularization parameter $\lambda > 0$, initial weight matrix $W\in\mathbb{R}^{CK\times D}$, 
number of rounds $T$, loss function $l(W,x,y)$}
\KwOut{Weight matrix $W\in\mathbb{R}^{C\times D}$}
\For{$t=1,2,\ldots,T$}{
  Draw example $(x,y)$ uniformly at random from $\mathcal{D}$\;
  $\eta \leftarrow \frac{1}{\lambda t}$\;
  $W \leftarrow (1- \lambda\eta) W$\;
  \If{ $l(W,x,y) > 0$}{
    $k \leftarrow \arg\max_{1\leq k\leq K} w_{y,k}^\top x$\;
    $c,k' \leftarrow \arg\max_{c\neq y,\; 1\leq k'\leq K}(w_{c,k'} - w_{y,k})^\top x$\;
    \tcc{Adjust the rows of $W$ contributing to the loss}
    $w_{c,k'} \leftarrow w_{c,y'} - \eta x$\;
    $w_{y,k} \leftarrow w_{y,k} + \eta x$\;
   }
  \tcc{Optional Projection Step}
  $\left( W \leftarrow \min\left\{1,\frac{1/\sqrt{\lambda}}{\|W\|}\right\} W \right)\;

}

\caption{Multiclass Multicomponent Pegasos}
\label{alg:pegasos-single-multiclass-multicomponent}
\end{algorithm}

\subsubsection{Experimental Test of the Algorithm}

We now attempt to run this on real data.  The first test is to
consider the performance on a two-component classifier.  For this
experiment rather than drawing randomly from the distribution
we will run through the dataset twice for various settings of the
$\lambda$ regularization parameter in order to properly choose the lambda parameter

\begin{table}
  \centering
  \begin{tabular}{| c | c |  c |}
    \hline
    \lambda & Use Projection  & Error Rate \\ \hline\hline
    27 & Yes & \input{exp/pegasos/leehon39_warm_train_pegasos_dev_10Ninit_2C_27lambda_Prj_error_rate.txt} \\ \hline
    27 & No & \input{exp/pegasos/leehon39_warm_train_pegasos_dev_10Ninit_2C_27lambda_NoPrj_error_rate.txt} \\ \hline
    9 & Yes & \input{exp/pegasos/leehon39_warm_train_pegasos_dev_10Ninit_2C_9lambda_Prj_error_rate.txt} \\ \hline
    9 & No & \input{exp/pegasos/leehon39_warm_train_pegasos_dev_10Ninit_2C_9lambda_NoPrj_error_rate.txt} \\ \hline
    3 & Yes & \input{exp/pegasos/leehon39_warm_train_pegasos_dev_10Ninit_2C_3lambda_Prj_error_rate.txt} \\ \hline
    3 & No & \input{exp/pegasos/leehon39_warm_train_pegasos_dev_10Ninit_2C_3lambda_NoPrj_error_rate.txt} \\ \hline
    1/3 & Yes & \input{exp/pegasos/leehon39_warm_train_pegasos_dev_10Ninit_2C_0.333333lambda_Prj_error_rate.txt} \\ \hline
    1/3 & No & \input{exp/pegasos/leehon39_warm_train_pegasos_dev_10Ninit_2C_0.333333lambda_NoPrj_error_rate.txt} \\ \hline
    1/9 & Yes & \input{exp/pegasos/leehon39_warm_train_pegasos_dev_10Ninit_2C_0.111111lambda_Prj_error_rate.txt} \\ \hline
    1/9 & No & \input{exp/pegasos/leehon39_warm_train_pegasos_dev_10Ninit_2C_0.111111lambda_NoPrj_error_rate.txt} \\ \hline
    1/27 & Yes & \input{exp/pegasos/leehon39_warm_train_pegasos_dev_10Ninit_2C_0.037037lambda_Prj_error_rate.txt} \\ \hline
    1/27 & No & \input{exp/pegasos/leehon39_warm_train_pegasos_dev_10Ninit_2C_0.037037lambda_NoPrj_error_rate.txt} \\ \hline
    1/81 & Yes & \input{exp/pegasos/leehon39_warm_train_pegasos_dev_10Ninit_2C_0.0123457lambda_Prj_error_rate.txt} \\ \hline
    1/81 & No & \input{exp/pegasos/leehon39_warm_train_pegasos_dev_10Ninit_2C_0.0123457lambda_NoPrj_error_rate.txt} \\ \hline
    1/243 & Yes & \input{exp/pegasos/leehon39_warm_train_pegasos_dev_10Ninit_2C_0.00411523lambda_Prj_error_rate.txt} \\ \hline
    1/243 & No & \input{exp/pegasos/leehon39_warm_train_pegasos_dev_10Ninit_2C_0.00411523lambda_NoPrj_error_rate.txt} \\ \hline
  \end{tabular}
  \caption{Error rates for various settings of $\lambda$ in the multicomponent Crammer-Singer Pegasos with and without projection run for $2T=280450$ iterations}
  \label{tab:2-component-pegasos-crammer-singer}
\end{table}


\subsection{Regularization}
  
We note that the regularization of the weight vectors in pairwise training
versus one-versus-rest multiclass training is very different.  The weight
vectors in pairwise training $w_{c,c'}$ correspond to differences
between the one-versus-rest weight vectors $w_c-w_{c'}$. Thus,
the penalty in pairwise estimation has a different meaning--we
suspect that this may be leading to the discrepancy in performance.
In a recent paper on Pegasos \cite{pegasos11} the authors note
that including the projection step did not change their results
very much so in this work we will experiment with the multiclass
Pegasos both with and without the projection step.

A projection step for the multiclass Pegasos that constrains
pairwise differences would be more involved.  Rather than working
on the weight matrix $W\in\mathbb{R}^{C\times D}$ the regularization
operates on a pairwise difference matrix
\begin{equation}\label{eq:pegasos-pairwise-regularize-matrix}
V\in\mathbb{R}^{ (K_1K_{-1}+K_2K_{-2}+\cdots+K_{C-1}K_{1-C}) \times D}\qquad v_{c,k,c',k'} = w_{c,k} - w_{c',k'}
\end{equation}
where $K_c$ is, as in \autoref{sec:mixture-svm-pegasos}, the
number of mixture components in class $c$ where $1\leq c\leq C$
and $K_{-c}=\sum_{i=1}^{C-c} K_{c+i}$ is the number of pairings between
components in class $c$ can have with classes $c+i$ where $i>0$,
by convention we let $K_{-C}=0$.  The definition in \autoref{eq:pegasos-pairwise-regularize-matrix}
can be straightforwardly applied to the standard multiclass
setup where $K_c=1$ for every $1\leq c\leq C$ so
\begin{equation}
K_1K_{-1}+K_2K_{-2}+\cdots+K_{C-1}K_{1-C} = C\choose 2
\end{equation}
and more generally $K_1K_{-1}+K_2K_{-2}+\cdots+K_{C-1}K_{1-C}$ is
precisely the number of pairings between components coming from
different classes.  The appropriate regularization for
$V$ is difficult to choose but a simple projection step that will
not change the detection decisions is to just use the
Frobenius norm which means the projection step is a
simple scaling of the entire vector. 

\subsubsection{Regularizing weight differences}

In the standard multiclass SVM setting \cite{Crammer01}
the loss is computed as
\begin{equation}
l(W,x,y) = \max\{0,1 -\max_{c\neq y} (w_c - w_y)^\top x\}
\end{equation}
so we are computing a maximum over differences between the weight
vectors.  In the case of the multi-component multiclass
SVM we use a saddlepoint
\begin{equation}
l(W,x,y) = \max\{0,1 -\max_{c\neq y\colon 1\leq k\leq K_c}\max_{1\leq k'\leq K_y} (w_{c,k} - w_{y,k'})^\top x\}
\end{equation}
so the relevant quantity for the prediction loss is a difference of weight vectors
rather than the weight vectors themselves--this reflects the fact that 
we care about the relative maxima over a set of predictors rather than
an actual fixed maximum. One consequence of this is that
$\|w_c\|^2$ may be large for every class $c$ but
$\|w_c-w_{c'}\|^2$ could be small if the vectors are highly correlated, i.e.
there are many irrelevant features that are shared across different weight
vectors.  Observe that 
\begin{equation}
\|w_c-w_{c'}\|^2 = \|w_c\|^2 + \|w_{c'}\|^2 - 2w_c^\top w_{c'}
\end{equation}
so that if $w_c$ and $w_{c'}$ have a high degree of negative correlation 
(and are very high dimensional) then
$\|w_c-w_{c'}\|^2$ could be much larger than $\|w_c\|^2 + \|w_{c'}\|^2$, which
would mean that small changes in a datum $\|x\|$ could cause very large
changes in the prediction which means that the SVM algorithm would
lose some of its robustness guarantee.  Additionally, if there are many
features in class $c$ that correlate with features in class $c'$
then we can allow $w_c$ and $w_{c'}$ to be large in those areas 
(perhaps allowing them to better discriminate against a third class
$c''$) without receiving a large penalty, in a certain sense large values
in the weight vectors $W$ will only be penalized if they contribute to
a variance in the prediction.  Correlated feature vectors in the multiclass
case essentially means you count information against yourself more than you'll
actually use it.
  The 
authors argue that it is impractical to use this regularization
however we respectfully disagree and optimization can be done
efficiently.

\subsection{Pegasos Regularizing Difference of Weight Vectors}

We consider the problem of regularizing the difference of
weight vectors rather than the weight vectors themselves
where the weight vectors have been given a warm start.  In this
section we will consider the multiple component case and
we will present the algorithm only for the case where we
see a single example used in the optimization.

For the purposes of a somewhat cleaner presentation we assume
that each class $c\in [C]$ has a fixed number $K$ of mixture 
components. Let $M=\frac{(CK)^2}{2}$ and $V\in \mathbb{R}^{M\times D}$ denote the matrix of weight
vector differences where $C$ is the number of phone class and
$K$ is the number of components per class so that
for $1\leq c\leq c'\leq C$ and $1\leq k, k'\leq C$ (we restrict
$k\leq k'$ in the case where $c=c'$) we have
a unique index $i(c,k,c',k')$ such that row $i(c,k,c',k')$ of
$V$ is defined as
\begin{equation}
  v_{c,k,c',k'} = w_{c,k} - w_{c',k'}.
\end{equation}
Formally we denote by $\mathcal{P}$ the set of all indices
$(c,k,c',k')$ satisfying 
\begin{align}
& 1\leq c\leq c' \leq C\\
& 1\leq k,k'\leq K\\
&\begin{cases}
 k< k' & c=c'\\
 \text{no restriction} & c\neq c'.
\end{cases}
\end{align}
Note that we could have defined $V$ using $M'= {CK\choose 2}$
which would eliminate the zero rows of $V$ corresponding to indices
$i(c,k,c,k)$.  Doing so would mean that $V$ would have fewer rows
but it would not change the norm $\|V\|$ since norms are invariant
to zero rows.  We find this definition of $V$ with zero rows
produces a simpler algorithm in the primal case.

The form of our algorithm is the same: minimize a loss and a norm
on $V$ (corresponding to a complexity constraint). The choice
of norm has a large impact on the algorithm and its not entirely
clear what the norm on $V$ should be.  One of the primary functions
of the norm is to improve robustness to small changes in the
data so that if $x\in\mathbb{R}$ then 
\begin{equation}
f(W,x) = \max_{c\neq c',k}\min_{c',k'}(w_{c,k}-w_{c',k'})^\top x 
\end{equation}
is our raw classifier output and we want to put a Lipshitz condition
on our classifier function so that
\begin{equation}
|\max_{c\neq c',k}\min_{c',k'}(w_{c,k}-w_{c',k'})^\top \delta | = |f(W,x) - f(w,x+\delta)|\leq \kappa \|\delta\|
\end{equation}
for some norm on $\delta$ thus we want a condition of the form
\begin{equation}\label{eq:lipschitz-multicomponent-svm}
\max_{\|\delta\|\leq 1} |\max_{c\neq c',k}\min_{c',k'}(w_{c,k}-w_{c',k'})^\top \delta | \leq \kappa.
\end{equation}
Observe that this is equivalent to the condition 
\begin{equation}
\max_{\|\delta\|\leq 1} |\max_{c\neq c',k}\min_{c',k'}(w_{c,k}-w_{c',k'})^\top \delta |^2 \leq \kappa^2
\end{equation}
and that
\begin{equation}
\max_{\|\delta\|\leq 1} |\max_{c\neq c',k}\min_{c',k'}(w_{c,k}-w_{c',k'})^\top \delta |^2 \leq \max_{\|\delta\|\leq 1} \sum_{c,k,c',k'} | (w_{c,k}-w_{c',k'})^\top \delta |^2 = \|V\|_2^2
\end{equation}
so that $\|V\|_2^2\leq \kappa^2$ guarantees
the condition in \autoref{eq:lipschitz-multicomponent-svm}. We will
consider this norm for the moment because of the ease with which
regularization is handled, however, we will consider other possibilities in the coming section.

The objective function is then
\begin{equation}
P_{all}(W) = \frac{\lambda}{2}\|V(W)\|^2 + \frac{1}{N}\sum_{n=1}^N l(W,x_n,y_n)
\end{equation}
where $l(W,x,y)$ is the multi-component multi-class hinge loss 
described in \autoref{eq:mixture_multiclass_hinge_loss}. In order
to compute the stochastic subgradient of $w_{c,k}$ we must compute
the partial derivative of $\|V\|_2^2$ taken with respect to
$w_{c,k}$
\begin{align}
\frac{\partial}{\partial w_{c,k}} \frac{1}{2}\|V\|_2^2 =&  \frac{1}{2}\sum_{(c',k',c'',k'')\in\mathcal{P}} \frac{\partial}{\partial w_{c,k}} \|w_{c',k'} -w_{c'',k''}\|^2\\
=& \frac{1}{2}\sum_{(c,k,c',k')\in\mathcal{P}} 2w_{c,k} -2w_{c',k'}\\
=& M w_{c,k} - \sum_{c',k'} w_{c',k'}
\end{align}
so that while $\|V\|_2^2$ may be complicated to explicitly compute
its partial derivative is very simple. The potential subgradients
are then
\begin{equation}
\partial_{w_{c,k}}P(W) = \begin{cases}
\lambda \left(Mw_{c,k} - \sum_{c',k'} w_{c,k}\right) - x & c=y,\; k=\arg\max_{k'} w_{y,k}^\top x\\
\lambda \left(Mw_{c,k} - \sum_{c',k'} w_{c,k}\right) + x & (c,k) = \underset{(c',k')\colon c\neq y}{\arg\max} w^\top x\\
\lambda \left(Mw_{c,k} - \sum_{c',k'} w_{c,k}\right) & \text{otherwise}.
\end{cases}
\end{equation}

We now need to work out the projection step of the algorithm. The
projection amounts to solving the problem
\begin{align}\label{eq:l2difference-projection-objective}
&\min_{U} \frac{1}{2}\|W - U\|^2_2\\
&\text{s.t.}\; \frac{1}{2}\|V(U)\|^2_2\leq \kappa.
\end{align}
The Lagrangian is then
\begin{equation}
\mathcal{L}(U,\lambda) = \frac{1}{2}\|W - U\|^2_2 + \lambda\left(\sum 
\frac{1}{2}\|u_{c,k} - u_{c',k'}\|^2_2 -\kappa\right)
\end{equation}
so that
\begin{align*}
\frac{\partial}{\partial u_{c,k}} \mathcal{L}(U,\lambda) =& u_{c,k}-w_{c,k}
 +\lambda \left( Mu_{c,k} - \sum_{c',k'} u_{c',k'} \right)\\
=& (1+\lambda(M-1))u_{c,k} - \lambda\sum_{(c',k')\neq (c,k)} u_{c',k'} -w_{c,k}.
\end{align*}
So that by the first order conditions
\begin{equation}
(1+\lambda(M-1))u_{c,k} - \lambda\sum_{(c',k')\neq (c,k)} u_{c',k'} =w_{c,k}
\end{equation}
which implies that 
\begin{equation}\label{eq:first_attemp_uck}
u_{c,k} = \frac{ w_{c,k} + \lambda \sum_{c',k'} w_{c',k'}}{ 1 + \lambda M }.
\end{equation}
We note that
\begin{align*}
\|V(\frac{1}{1+\lambda M}(W + \lambda W_{sums}))\|^2
=& \sum_{(c,k,c',k')\in\mathcal{P}} \|u_{c,k} - u_{c',k'} \|^2\\
=& \sum_{(c,k,c',k')\in\mathcal{P}} \frac{1}{(1+\lambda M)^2}\|w_{c,k} - w_{c',k'} \|^2\\
=& \frac{1}{(1+\lambda M)^2} \|V(W)\|^2
\end{align*}
so the condition that
\begin{equation}
\frac{1}{2}\|V(U)\|^2 \leq \kappa
\end{equation}
is equivalent to
\begin{equation}\label{eq:difference-projection-lambda}
\frac{1}{2(1+\lambda M)^2} \|V(W)\|^2 \leq \kappa.
\end{equation}
Clearly, if $\|V(W)\|^2 \leq 2 \kappa$ then the objective in \autoref{eq:l2difference-projection-objective}
 (which is always non-negative)
is minimized with $U=W$ since the value of the problem would be
zero and in that case $\lambda =0$. Otherwise if $\|V(W)\|^2 > 2\kappa$ then clearly $\lambda>0$
to satisfy \autoref{eq:difference-projection-lambda}
so by complementary slackness 
\begin{equation}
\frac{1}{2(1+\lambda M)^2} \|V(W)\|^2 = \frac{1}{2}\|V(U)\|^2 = \kappa
\end{eqation}
so 
\begin{equation}
\lambda = \sqrt{\frac{\|V(W)\|^2}{2\kappa M^2}}-\frac{1}{M}
\end{equation}
which means that \autoref{eq:first_attemp_uck}
we get
\begin{equation}\label{eq:u_ck-projection-value}
u_{c,k} = \frac{\sqrt{2\kappa}}{\|V(W)\|}\left[w_{c,k} + \left(\sqrt{\frac{\|V(W)\|^2}{2\kappa M^2}}-\frac{1}{M}\right) \sum_{c',k'} w_{c',k'}\right].
\end{equation}
Having worked out the projection step we can now summarize the
steps in \autoref{alg:pegasos-single-multiclass-multicomponent-diff-regularized}.
\begin{algorithm}[t]
\SetAlgoNoLine
\KwIn{Data set $\mathcal{D}=\{(x,y)\}\subset \mathbb{R}^D\times [C]$, regularization parameter $\lambda > 0$, initial weight matrix $W\in\mathbb{R}^{CK\times D}$, 
number of rounds $T$, loss function $l(W,x,y)$, weight difference function $V$}
\KwOut{Weight matrix $W\in\mathbb{R}^{C\times D}$}
\For{$t=1,2,\ldots,T$}{
  Draw example $(x,y)$ uniformly at random from $\mathcal{D}$\;
  $\eta \leftarrow \frac{1}{\lambda t}$\;
\tcc{Denote by $\mathbf{1}$ the vector of all ones}
  $W \leftarrow W - \lambda\eta \left(MW - \sum_{c,k} w_{c,k}\mathbf{1}^\top\right)$\;
  \If{ $l(W,x,y) > 0$}{
    $k \leftarrow \arg\max_{1\leq k\leq K} w_{y,k}^\top x$\;
    $c,k' \leftarrow \arg\max_{c\neq y,\; 1\leq k'\leq K}(w_{c,k'} - w_{y,k})^\top x$\;
    \tcc{Adjust the rows of $W$ contributing to the loss}
    $w_{c,k'} \leftarrow w_{c,y'} - \eta x$\;
    $w_{y,k} \leftarrow w_{y,k} + \eta x$\;
   }
  \tcc{Optional Projection Step}
  \If{ $\|V(W)\|^2 > \frac{1}{\lambda}$}{
    $W = \frac{1}{\|V(W)\|\sqrt{\lambda}}W + \frac{\|V(W)\|\sqrt{\lambda}-1}{M\|V(W)\|\sqrt{\lambda}} \sum_{c,k} w_{c,k}\mathbf{1}^\top $

    }
}

\caption{Multiclass Multicomponent Weight Difference Regularized Pegasos}
\label{alg:pegasos-single-multiclass-multicomponent-diff-regularized}
\end{algorithm}



\subsection{Likelihood Model}
\begin{enumerate}
\item the log-odds ratio interpretation of weight vectors
\item the difference for different types of regularization and modeling context 
\end{enumerate}

Another ambiguity is what to do in the case where the two
best models for the data all come from a common class so that
$w_{y,k}^\top x$ and $w_{y,k'}^\top x$ are the largest scores, in that
case we only consider the larger, $w_{y,k_*}^\top x$, and compare
it to the scores from the other classes, thus, each examp


\subsection{Max-min Regularization}

\subsection{Binary Regularization}


\section{Mixture of Multivariate Bernoulli Models with Pegasos Training}


\section{Parts Model Versus Multivariate Bernoulli Models}


\section{Hand-designed Parts for Tracking Formants}

\section{Subspace Mixture of Bernoullis}

We extend the model from Povey to mixtures of multivariate Bernoullis

\section{Comparing Generative and Discriminative classifiers}

%%%
%
% TODO!!!!!!!!!!!!!
%
Things to discuss
\begin{enumerate}
\item how performace differs in terms of average performance per phone class
\item how the weight vectors differ in terms of the supports
\item show the difference in what is driving detection decisions between the two
\end{enumerate}

One striking difference between generative and discriminative
classifiers is that discriminative classifiers trained with
regularization tend to zero out coordinates that are not useful
for classification.  This can be seen in the following plots:
short syllables

\bibliographystyle{plain}
\bibliography{../bibliography}


\end{document}
