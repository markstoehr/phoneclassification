% ---- ETD Document Class and Useful Packages ---- %
\documentclass{article}
\usepackage{subfigure,epsfig,amsfonts,bigints}
\usepackage{natbib}
\usepackage{amsmath}
\usepackage{amssymb}
\usepackage{amsthm}
\usepackage{algorithmic,algorithm}
\usepackage{caption}
\usepackage{hyperref}
\usepackage{fullpage}

%% Use these commands to set biographic information for the title page:
\title{Phone Classification Work}
\author{Mark Stoehr}
\date{\today}

\renewcommand{\algorithmicrequire}{\textbf{Input:}}
\renewcommand{\algorithmicensure}{\textbf{Output:}}

\begin{document}

\maketitle

We run a variety of experiments for classifying phones with different feature sets.
\section{aa versus aw}

The experiment is carried out by first extracting the training, development, and test data.  We make use of the
experimental setup contained in the template speech recognition experiments.  

\subsection{Data Preparation}

To extract the data we first add the directory that will contain the files
\begin{verbatim}
export PYTHONPATH=$PYTHONPATH:/home/mark/Template-Speech-Recognition/Development/051913
export PATH=$PATH:/home/mark/Template-Speech-Recognition/Development/051913
\end{verbatim}
and we construct files that include the paths to the data.
We do this via using a split between the training and test given
by TIMIT. We also divide the training into a model training portion
and a development tuning set.  We exclude the \texttt{sa} sentences.
\begin{verbatim}
TIMIT=/home/mark/Template-Speech-Recognition/Development/051913
dir=`pwd`/data/local/data
mkdir -p $dir
local=`pwd`/local
utils=`pwd`/utils
conf=`pwd`/conf

export KALDI_ROOT=/home/mark/Software/kaldi-trunk
export PATH=$PWD/utils/:$KALDI_ROOT/src/bin:$KALDI_ROOT/tools/openfst/bin:$KALDI_ROOT/tools/irstlm/bin/:$KALDI_ROOT/src/fstbin/:$KALDI_ROOT/src/gmmbin/:$KALDI_ROOT/src/featbin/:$KALDI_ROOT/src/lm/:$KALDI_ROOT/src/sgmmbin/:$KALDI_ROOT/src/sgmm2bin/:$KALDI_ROOT/src/fgmmbin/:$KALDI_ROOT/src/latbin/:$KALDI_ROOT/src/nnetbin:$KALDI_ROOT/src/nnet-cpubin/:$KALDI_ROOT/src/kwsbin:$PWD:$PATH
export LC_ALL=C

export PATH=$PATH:$KALDI_ROOT/tools/irstlm/bin
sph2pipe=$KALDI_ROOT/tools/sph2pipe_v2.5/sph2pipe
\end{verbatim}
and these make sure that we have all the files setup to 
perform the data extraction and get prepped for the experiment.
The next step is to go into the directory and setup a soft-linked
version of TIMIT
\begin{verbatim}
cd $dir

# Make directory of links to the TIMIT disk.  This relies on the command
# line arguments being absolute pathnames.
rm -r links/ 2>/dev/null
mkdir links/

ln -s $TIMIT links
\end{verbatim}
and then we get all the files that say where the data is contained
\begin{verbatim}
TrainDir=$TIMIT/train
find -L $TrainDir \( -iname 's[ix]*.WAV' -o -iname 's[ix]*.wav' \) > train.flst
$local/flist_to_scp.py < train.flst | sort > train_wav.scp
cat train_wav.scp | awk '{print $1}' > train.uttids
cat train_wav.scp | sed 's/wav/phn/'> train_phn.scp
awk '{print $2}' train_wav.scp > train.wav
awk '{print $2}' train_phn.scp > train.phn
# awk '{printf("%s '$sph2pipe' -f wav %s\n", $1, $2);}' < train_dev_sph.scp > train_dev_wav.scp
\end{verbatim}
and we can assume that the wav files (as actual wav files) are already
present in the directory structure since we did this conversion
earlier on that directory.  This is the complete training set
as outline in Halberstadt.  The next step is 
to get the development and test datasets.
We first output the lists of speakers, as with the training set
we will exclude the SA sentences
\begin{verbatim}
cat << "EOF" | sort > $conf/dev_spk
faks0
mmdb1
mbdg0
fedw0
mtdt0
fsem0
mdvc0
mrjm4
mjsw0
mteb0
fdac1
mmdm2
mbwm0
mgjf0
mthc0
mbns0
mers0
fcal1
mreb0
mjfc0
fjem0
mpdf0
mcsh0
mglb0
mwjg0
mmjr0
fmah0
mmwh0
fgjd0
mrjr0
mgwt0
fcmh0
fadg0
mrtk0
fnmr0
mdls0
fdrw0
fjsj0
fjmg0
fmml0
mjar0
fkms0
fdms0
mtaa0
frew0
mdlf0
mrcs0
majc0
mroa0
mrws1
EOF
\end{verbatim}
and the test speakers are
\begin{verbatim}
cat << "EOF" | sort > $conf/core_test_spk
mdab0
mwbt0
felc0
mtas1
mwew0
fpas0
mjmp0
mlnt0
fpkt0
mlll0
mtls0
fjlm0
mbpm0
mklt0
fnlp0
mcmj0
mjdh0
fmgd0
mgrt0
mnjm0
fdhc0
mjln0
mpam0
fmld0
EOF
\end{verbatim}
and setup those configuration files we then get the development set
and core test set data list files
\begin{verbatim}
TestDir=$TIMIT/test
find -L $TestDir \( -iname 's[ix]*.WAV' -o -iname 's[ix]*.wav' \) > test.flst
$local/flist_to_scp.py < test.flst | sort > test_wav.scp
cat test_wav.scp | awk '{print $1}' > test.uttids
cat test_wav.scp | sed 's/wav/phn/'> test_phn.scp
awk '{print $2}' test_wav.scp > test.wav
awk '{print $2}' test_phn.scp > test.phn
\end{verbatim}
and then we use those configuration files to devide between
\texttt{dev} and \texttt{core} which refer to the development set
and core test set, respectively.
\begin{verbatim}
$local/extract_by_spk.py -i test_wav.scp --spk $conf/dev_spk > dev_wav.scp
$local/extract_by_spk.py -i test_wav.scp --spk $conf/core_test_spk > core_test_wav.scp
$local/extract_by_spk.py -i test_phn.scp --spk $conf/dev_spk > dev_phn.scp
$local/extract_by_spk.py -i test_phn.scp --spk $conf/core_test_spk > core_test_phn.scp

awk '{print $1}' dev_wav.scp > dev.uttids
awk '{print $1}' core_test_wav.scp > core_test.uttids

awk '{print $2}' dev_wav.scp > dev.wav
awk '{print $2}' dev_phn.scp > dev.phn
awk '{print $2}' core_test_wav.scp > core_test.wav
awk '{print $2}' core_test_phn.scp > core_test.phn
\end{verbatim}
and now we have setup the files for the experiment.

\subsection{Data Extraction}

We use some tools developed for other experiments to get the
files that contain the instances of the data.
\begin{verbatim}
paste train.phn train.wav | CObjectInstances.py --label_sequence aa > aa_train.trans
paste train.phn train.wav | CObjectInstances.py --label_sequence aw > aw_train.trans
\end{verbatim}
we want to extract large enough windows that we get all the examples
within the window, so, in order to do that we look at the 
statistics of the lengths of the examples.  The tool we use
is \texttt{utils/length\_stats.py}
\begin{verbatim}
$utils/length_stats.py -i aa_train.trans --out_lengths aa_train.lengths \
   --out_hist aa_train_lengths.png --nbins 50
$utils/length_stats.py -i aw_train.trans --out_lengths aw_train.lengths \
   --out_hist aw_train_lengths.png --nbins 50

maxlength=`cat *.lengths | sort -n | tail -1`
\end{verbatim}
now that we have the max length we will extract all of the examplesc
based on the config file.
\begin{verbatim}
cat << "EOF" > $conf/main.config
[SPECTROGRAM]
sample_rate=16000
num_window_samples=320
num_window_step_samples=80
fft_length=512
kernel_length=7
freq_cutoff=3000
preemphasis=.95
use_mel=False
do_mfccs=False
no_use_dpss=False
mel_nbands=40
num_ceps=13
liftering=.6
include_energy=False
include_deltas=False
include_double_deltas=False
delta_window=9
do_freq_smoothing=False
mel_smoothing_kernel=-1

[SVM]
example_length=.2
kernel='linear'
penalty_list=['unreg', '1.0',
                                                 'little_reg','0.1',
                                                 'reg', '0.05',
                                                 'reg_plus', '0.01',
                                                 'reg_plus_plus','0.001']


[EDGES]
block_length=40
spread_length=1
threshold=.7
magnitude_block_length=0
abst_threshold=(0.025,  0.025,  0.015,  0.015,  0.02 ,  0.02 ,  0.02 ,  0.02 )
magnitude_threshold=.4
magnitude_spread=1
magnitude_and_edge_features=False
magnitude_features=False
mag_smooth_freq=0
mag_downsample_freq=0
auxiliary_data=False
auxiliary_threshold=.5
num_mag_channels=10
num_axiliary_data=3
save_parts_S=False
EOF
\end{verbatim}
an important component of this is the
\texttt{example\_length} under \texttt{[SVM]} which is the number
of milliseconds that we want to capture the phone for
and then we have the subtle issue of timing arise since the statistics of the data are computed over frames which are subsampled
from the data and this needs to be considered in order to compute
the actual segmens that we are going to extract.  However, we are
basing this experiment on the work of Joakim anden and they used
a $T=200ms$ time which corresponds to $38$ frames as given
the $80$ sample subsample factor for the frames (where we only count to the middle of the last frame).  So we center
the frames as best we can given the input data.  Namely we wish to compute the squared distance over all the frames for the extracted data from the middle sample of the example phone and pick the set
of frames that minimize that distance.

The formalu is that the interval chosen will have the form
$$ [80t, 80t+38*80+160)  $$
which may be written more generally
$$ [ht,ht+N] $$
where $h$ is the subsample factor for the feature vectors and
$N$ is the window length in samples for what frames will be used in
the object.  
The actual example will be in an interval $[a,b]$
we wish to minimize the sum of squared distances to 
$$ \frac{a+b}{2}$$
in the samples of the given interval
given a choice of an integer $t$.  The cost $C(t)$ for a given choice
of $t$ is
\begin{equation}\begin{array}{rcl}
C(t) &=& \sum_{\tau=0}^{N-1} \left(ht+\tau-\frac{a+b}{2}\right)^2\\
&=& \sum_{\tau=0}^{N-1} (ht+\tau)^2-2(ht+\tau)\frac{a+b}{2} +\left(\frac{a+b}{2}\right)^2 
\end{array}\end{equation}
so that the equivalent optimization is
\begin{equation}
\arg\min_t C(t) = \arg\min_t \sum_{\tau=0}^{N-1} (ht+\tau)^2-2(ht+\tau)\frac{a+b}{2}
\end{equation}
so taking the derivative and equating to zero we get
\begin{equation}\begin{array}{rcl}
\sum_{\tau=0}^{N-1} 2h(ht+\tau)- 2h\frac{a+b}{2} &=& 0\\
Nht &=& N\frac{a+b}{2}-\frac{N(N-1)}{2}\\
t &=& \frac{a+b-N+1}{2h}
\end{array}
\end{equation}
and then will just let $t$ be the integer that is closest to that
value.

The code that implements this can then output the start and end
frames relative to the settings given by \texttt{main.config}
\begin{verbatim}
$local/sampletrans2frametrans.py -c $conf/main.config --transcript aa_train.trans\
  > aa_train.frame_trans
$local/sampletrans2frametrans.py -c $conf/main.config --transcript aw_train.trans\
  > aw_train.frame_trans
\end{verbatim}
and having developed those we may then actually do the extraction
of the data matrices
\begin{verbatim}
$local/extract_data_matrix.py -c $conf/main.config --transcript aa_train.frame_trans\
  --uttids train.uttids\
  --output aa_train_examples.npy\
  --output_spec aa_train_examples_S.npy\
  --output_info aa_train_info.npy
$local/extract_data_matrix.py -c $conf/main.config --transcript aw_train.frame_trans\
  --uttids train.uttids\
  --output aw_train_examples.npy\
  --output_spec aw_train_examples_S.npy\
  --output_info aw_train_info.npy
\end{verbatim}
which completes the data extraction step for the training data. The files that were created are
\begin{verbatim}
[aa,aw]_train_examples.npy
[aa,aw]_train_examples_S.npy
[aa,aw]_train_info.npy
\end{verbatim}
where the first is all the examples in a matrix for the binary
features, \texttt{aa\_train\_examples\_S.npy} contains the spectrogram
features, and
\texttt{aa\_train\_info.npy} is a four column array where each 
row corresponds to an example, the first column is the utterance
id (which is the row of the utterance in \texttt{train.uttids}),
the second column is the order of the examples sorted by time in the utterance
(the first occurrence within the utterance is zero,
second occurrence is one, etc.), and the last two columns are the
start and end frames for the extracted examples.

The extraction for the test data runs as follows
\begin{verbatim}
for pref in dev core_test ; do
    paste ${pref}.phn ${pref}.wav | CObjectInstances.py --label_sequence aa\
      > aa_${pref}.trans
    paste ${pref}.phn ${pref}.wav | CObjectInstances.py --label_sequence aw\
      > aw_${pref}.trans

    $local/sampletrans2frametrans.py -c $conf/main.config --transcript aa_${pref}.trans\
      > aa_${pref}.frame_trans
    $local/sampletrans2frametrans.py -c $conf/main.config --transcript aw_${pref}.trans\
      > aw_${pref}.frame_trans

    $local/extract_data_matrix.py -c $conf/main.config\
      --transcript aa_${pref}.frame_trans\
      --uttids ${pref}.uttids\
      --output aa_${pref}_examples.npy\
      --output_spec aa_${pref}_examples_S.npy\
      --output_info aa_${pref}_info.npy
    $local/extract_data_matrix.py -c $conf/main.config\
      --transcript aw_${pref}.frame_trans\
      --uttids ${pref}.uttids\
      --output aw_${pref}_examples.npy\
      --output_spec aw_${pref}_examples_S.npy\
      --output_info aw_${pref}_info.npy

done
\end{verbatim}
whcih completes the extraction.  The \texttt{--output} flat on the
call to \texttt{\$local/extract\_data\_matrix.py} indicates
where the data is saved to.

\section{Training the SVM}

We first create an experiment directory
\begin{verbatim}
cd ../../../
exp_dir=exp/aw_aa_exp
mkdir -p $exp_dir
datadir=$dir
\end{verbatim}
and then we train the svm on the extracted examples
under a variety of settings (as spelled out in \texttt{conf/main.config}).  The command to do this is
\begin{verbatim}
CSVMTrain.py --input_data_matrices $datadir/aa_train_examples.npy\
    $datadir/aw_train_examples.npy\
    --input_test_data_matrices $datadir/aa_dev_examples.npy\
    $datadir/aw_dev_examples.npy --output_fls_prefix $exp_dir/svm_train\
    --config $conf/main.config -v
\end{verbatim}
and this outputs a set of svm filters and results for the development
data set (we wait to do the core test set experiment until after.
The lowest error rate we obtained was $12.12\%$ which gets very
low error on \texttt{aa} but $44\%$ error on \texttt{aw}.

\section{Extract matrices for all data}

Given the success of running that first experiment we now collect
data for all the examples. This means that the procedure outlined above is generalized.  The first step to do this is to get the list
of phones, and also get the phone-mappings where we map down to 48 phone classes from the complete 61 classes (and exclude \texttt{q}).
We do
\begin{verbatim}
cat << "EOF" > $conf/phones.60-48
aa aa
ae ae
ah ah
ao ao
aw aw
ax ax
ax-h ax
axr er
ay ay
b b
bcl vcl
ch ch
d d
dcl vcl
dh dh
dx dx
eh eh
el el
em m
en en
eng ng
epi epi
er er
ey ey
f f
g g
gcl vcl
h# sil
hh hh
hv hh
ih ih
ix ix
iy iy
jh jh
k k
kcl cl
l l
m m
n n
ng ng
nx n
ow ow
oy oy
p p
pau sil
pcl cl
r r
s s
sh sh
t t
tcl cl
th th
uh uh
uw uw
ux uw
v v
w w
y y
z z
zh zh
EOF
\end{verbatim}
and for testing we do
\begin{verbatim}
cat << "EOF" > $conf/phones.48-39
aa aa
ae ae
ah ah
ao aa
aw aw
ax ah
ay ay
b b
ch ch
cl sil
d d
dh dh
dx dx
eh eh
el l
en n
epi sil
er er
ey ey
f f
g g
hh hh
ih ih
ix ih
iy iy
jh jh
k k
l l
m m
n n
ng ng
ow ow
oy oy
p p
r r
s s
sh sh
sil sil
t t
th th
uh uh
uw uw
v v
vcl sil
w w
y y
z z
zh sh
EOF
\end{verbatim}
and then  we get the list of phone classes that we are going to train
as
\begin{verbatim}
awk '{ print $2 }' $conf/phones.60-48 | sort | uniq > $datadir/phone.list
\end{verbatim}
which gives us the set of phones that we can loop through to extract
all of the relevant training data.
We then do
\begin{verbatim}
cd $datadir
cat $datadir/phone.list | while read phn   ; do
    echo ${phn}
    for data_set in train dev core_test ; do
        echo ${data_set}
        paste ${data_set}.phn ${data_set}.wav | CObjectInstances.py\
            --label_sequence $phn\
            --phone_reduction_file $conf/phones.60-48\
            > ${phn}_${data_set}.trans

        $utils/length_stats.py -i ${phn}_${data_set}.trans\
            --uttids ${data_set}.uttids\
            --out_lengths ${phn}_${data_set}.lengths \
            --out_hist ${phn}_${data_set}_lengths.png --nbins 50

        $local/sampletrans2frametrans.py -c $conf/main.config\
            --transcript ${phn}_${data_set}.trans\
            > ${phn}_${data_set}.frame_trans

        $local/extract_data_matrix.py -c $conf/main.config --transcript ${phn}_${data_set}.frame_trans\
            --uttids ${data_set}.uttids\
            --output ${phn}_${data_set}_examples.npy\
            --output_spec ${phn}_${data_set}_examples_S.npy\
            --output_info ${phn}_${data_set}_info.npy
    done
done
\end{verbatim}
and then we verify that we produced all of the features that
we think we produced
\begin{verbatim}
rm -f all_data_matrix_list
touch all_data_matrix_list
cd $datadir
cat $datadir/phone.list | while read phn   ; do
    echo ${phn}
    for data_set in train dev core_test ; do
        echo ${phn}_${data_set}_examples.npy >> all_data_matrix_list
        echo ${phn}_${data_set}_examples_S.npy >> all_data_matrix_list
        echo ${phn}_${data_set}_info.npy >> all_data_matrix_list
    done
done

sort all_data_matrix_list | uniq > tmp
cp tmp all_data_matrix_list

ls -1 | sort | uniq > file_list
\end{verbatim}

another part of the standard svm setup is that we want the lengths
and the example lengths which were computed by the file \texttt{\$utils/length\_stats.py} and saved to the files of the form \texttt{\$\{phn\}\_\$\{data\_set\}.lengths} which have a three column format
where the first column is the numeric utterance index (which is the
order of the utterance in the \texttt{\$\{data\_set\}.uttids} file),
the order the example came in, and the length in samples.


\section{Training Models}

We use two approaches, the first is to just use the SVM which 
we experiment with right now.  And we also supplement the 
decision function with the length information.

\subsection{SVM training}

\begin{verbatim}
cd ../../../
exp_dir=exp/all_phones_exp
mkdir -p $exp_dir
datadir=$dir
\end{verbatim}
and then we train the svm on the extracted examples
under a variety of settings (as spelled out in \texttt{conf/main.config}).  The command to do the trainging (it also does a little
testing) is
\begin{verbatim}
CSVMTrain.py --input_data_matrices $datadir/aa_train_examples.npy\
    $datadir/aw_train_examples.npy\
    --input_test_data_matrices $datadir/aa_dev_examples.npy\
    $datadir/aw_dev_examples.npy --output_fls_prefix $exp_dir/svm_train\
    --config $conf/main.config -v
\end{verbatim}
and after we run that over all pairs of examples as follows
\begin{verbatim}
set -- `cat $datadir/phone.list`
for phn1 ; do
    shift
    for phn2 ; do
        echo "$phn1 against $phn2"
        CSVMTrain.py --input_data_matrices\
            $datadir/${phn1}_train_examples.npy\
            $datadir/${phn2}_train_examples.npy\
            --input_test_data_matrices\
            $datadir/${phn1}_dev_examples.npy\
            $datadir/${phn2}_dev_examples.npy\
            --input_lengths\
            $datadir/${phn1}_train.lengths\
            $datadir/${phn2}_train.lengths\
            --input_test_lengths\
            $datadir/${phn1}_dev.lengths\
            $datadir/${phn2}_dev.lengths\
            --output_fls_prefix $exp_dir/svm_train_${phn1}_${phn2}\
            --config $conf/main.config -v
    done
done
\end{verbatim}
the bash portion of this is a little tricky -- we get bash to
make the phone list into the input arguments and shift through
them.

\begin{verbatim}
datadir=data/local/data
exp_dir=exp/all_phones_exp
commands_fl=$exp_dir/commands_fl
rm -f $commands_fl
set -- `cat $datadir/phone.list`
for phn1 ; do
    shift
    for phn2 ; do
        echo "$phn1 against $phn2"
        echo CSVMTrain.py --input_data_matrices\
            $datadir/${phn1}_train_examples.npy\
            $datadir/${phn2}_train_examples.npy\
            --input_test_data_matrices\
            $datadir/${phn1}_dev_examples.npy\
            $datadir/${phn2}_dev_examples.npy\
            --input_lengths\
            $datadir/${phn1}_train.lengths\
            $datadir/${phn2}_train.lengths\
            --input_test_lengths\
            $datadir/${phn1}_dev.lengths\
            $datadir/${phn2}_dev.lengths\
            --output_fls_prefix $exp_dir/svm_train_${phn1}_${phn2}\
            --config $conf/main.config -v >> $commands_fl
    done
done
\end{verbatim}
which will save all of the commands to
\texttt{commands_fl}
and then we add parallel to the commands
\begin{verbatim}
parallel=~/bin/parallel
\end{verbatim}
and then we can write the commands
\begin{verbatim}
$parallel --semaphore --jobs 6  --joblog $exp_dir/parallel.log -a $commands_fl
\end{verbatim}
and we also need to check how many have been completed at a given
time just in case the jobs get interrupted
we do this with
\begin{verbatim}
last_commands_fl=$exp_dir/last_commands_fl
local/check_completed.py < $commands_fl > $last_commands_fl
\end{verbatim}
and then we run it with
\begin{verbatim}
$parallel --semaphore --jobs 6  --joblog $exp_dir/parallel.log -a $last_commands_fl
\end{verbatim}
\begin{verbatim}
cat $last_commands_fl | while read line; do
sem --jobs 6 $line
done

\end{verbatim}
We construct a file with all of those commands listed in it 

\subsubsection{Scoring}


\begin{verbatim}
datadir=data/local/data
exp_dir=exp/all_phones_exp
phn_pairs=''
coef_fls=''
intercept_fls=''
penalty=reg_plus_plus
set -- `cat $datadir/phone.list`
for phn1 ; do
    shift
    for phn2 ; do
        phn_pairs="$phn_pairs $phn1 $phn2"
	coef_fls="$coef_fls $exp_dir/svm_train_${phn1}_${phn2}_linear_${penalty}_coef.npy"
        intercept_fls="$intercept_fls $exp_dir/svm_train_${phn1}_${phn2}_linear_${penalty}_intercept.npy"
    done
done
\end{verbatim}
and then we get
\begin{verbatim}
penalties="little_reg reg_plus reg_plus_plus reg_plus_plus_plus"
for penalty in $penalties; do
python local/CCollectSVMs.py --coefs $coef_fls --intercepts $intercept_fls --out_coefs $exp_dir/svm_train_all_pairs_linear_${penalty}_coefs.npy --out_intercepts $exp_dir/svm_train_all_pairs_linear_${penalty}_intercepts.npy --label_pairs $phn_pairs --out_ids $exp_dir/svm_train_all_pairs_${penalty}.ids --phns_to_ids $exp_dir/svm_train_all_pairs_${penalty}_phns.ids
done

\end{verbatim}
and then we are ready to test these on a given test set and get the scoring.
\begin{verbatim}
dev_examples=""
dev_lengths=""
phn_ids=""
phn_id=0
for phn in `cat $datadir/phone.list` ; do
  dev_examples="$dev_examples data/local/data/${phn}_dev_examples.npy"
  dev_lengths="$dev_lengths data/local/data/${phn}_dev.lengths"
  phn_ids="$phn_ids $phn_id"
  phn_id=$(( $phn_id + 1 ))
done

for penalty in $penalties ; do
python local/CRunSVM.py \
  --coefs $exp_dir/svm_train_all_pairs_linear_${penalty}_coefs.npy\
  --intercepts $exp_dir/svm_train_all_pairs_linear_${penalty}_intercepts.npy\
  --out_ids $exp_dir/svm_train_all_pairs_${penalty}.ids\
  --input_data_matrices $dev_examples\
  --input_lengths $dev_lengths\
  --out_confusion_matrix $exp_dir/svm_all_pairs_${penalty}_confusion.npy\
  --out_leehon_confusion_matrix $exp_dir/svm_all_pairs_${penalty}_leehon_confusion.npy\
  --input_ids $phn_ids\
  --leehon_mapping conf/phones.48-39\
  --phn_ids $exp_dir/svm_train_all_pairs_${penalty}_phns.ids
done

\end{verbatim}

\subsection{Mixture Model training}

We then have commands setup to write
\begin{verbatim}
mkdir -p exp/bernoulli_all_phones
exp_dir=exp/bernoulli_all_phones
data=data/local/data
for phn in `cat ${data}/phone.list` ; do
   echo phn=${phn}
   python local/CBernoulliMMTrain.py --config conf/main.config \
      --input_data_matrix ${data}/${phn}_train_examples.npy \
      --templates ${exp_dir}/${phn}_templates.npy \
      --weights ${exp_dir}/${phn}_weights.npy \
      --underlying_data ${data}/${phn}_train_examples_S.npy \
      --spec_templates ${exp_dir}/${phn}_S_templates.npy \
      --viz_spec_templates ${exp_dir}/${phn}_S_templates -v
done

for phn in `cat ${data}/phone.list` ; do
   echo phn=${phn}
   python local/check_components_nondegenerate.py --config conf/main.config \
      --input_data_matrix ${data}/${phn}_train_examples.npy \
      --templates ${exp_dir}/${phn}_templates.npy \
      --weights ${exp_dir}/${phn}_weights.npy \
      --underlying_data ${data}/${phn}_train_examples_S.npy \
      --spec_templates ${exp_dir}/${phn}_S_templates.npy \
      --viz_spec_templates ${exp_dir}/${phn}_S_templates -v
done




\end{verbatim}

and then we collect together the various models so that way
we can score

\begin{verbatim}
data=data/local/data
exp_dir=exp/bernoulli_all_phones
dev_examples=""
train_examples=""
templates=""
for phn in `cat ${data}/phone.list` ; do
  dev_examples="$dev_examples ${data}/${phn}_dev_examples.npy"
  train_examples="$train_examples ${data}/${phn}_train_examples.npy"
  templates="$templates ${exp_dir}/${phn}_templates.npy"
done

python local/CTestBernoulli.py --data $dev_examples --templates $templates\
    --out_scores ${exp_dir}/bernoulli_dev_scores.npy\
    --out_labels ${exp_dir}/bernoulli_dev_labels.npy\
    --out_components ${exp_dir}/bernoulli_dev_components.npy\
    --out_top_predicted_labels ${exp_dir}/bernoulli_dev_top_predicted_labels.npy\
    --out_top_predicted_components ${exp_dir}/bernoulli_dev_top_predicted_components.npy

python local/CTestBernoulli.py --data $train_examples --templates $templates\
    --out_scores ${exp_dir}/bernoulli_train_scores.npy\
    --out_labels ${exp_dir}/bernoulli_train_labels.npy\
    --out_components ${exp_dir}/bernoulli_train_components.npy\
    --out_top_predicted_labels ${exp_dir}/bernoulli_train_top_predicted_labels.npy\
    --out_top_predicted_components ${exp_dir}/bernoulli_train_top_predicted_components.npy


\end{verbatim}
these two give us a sense of the performance on the training and
testing data of the two models.  The results show that we get
a 58\% accuracy using the edges on the training data and a
54\% accuracy using the edges on the testing data. This indicates
that the two data sets are actually very similar.

\subsection{Combining the SVM with Bernoulli mixtures}
However, one statistic that we take note of is that most
of the true labels appear in the top few predicted labels. 
Hence it might make sense to run the SVM voting code but only
over the top predicted labels and see what the results are.

We run the following code to test that idea
\begin{verbatim}
exp_dir=exp/all_phones_exp
exp_dir_bernoulli=exp/bernoulli_all_phones
dev_examples=""
dev_lengths=""
phn_ids=""
phn_id=0
for phn in `cat $data/phone.list` ; do
  dev_examples="$dev_examples data/local/data/${phn}_dev_examples.npy"
  dev_lengths="$dev_lengths data/local/data/${phn}_dev.lengths"
  phn_ids="$phn_ids $phn_id"
  phn_id=$(( $phn_id + 1 ))
done

#for penalty in $penalties ; do
penalty=reg_plus_plus
python local/CRunSVMTopVotes.py \
  --coefs $exp_dir/svm_train_all_pairs_linear_${penalty}_coefs.npy\
  --intercepts $exp_dir/svm_train_all_pairs_linear_${penalty}_intercepts.npy\
  --out_ids $exp_dir/svm_train_all_pairs_${penalty}.ids\
  --input_data_matrices $dev_examples\
  --input_lengths $dev_lengths\
  --out_confusion_matrix $exp_dir_bernoulli/svm_all_pairs_top_bernouli_votes_${penalty}_confusion.npy\
  --out_leehon_confusion_matrix $exp_dir_bernoulli/svm_all_pairs_top_bernouli_votes_${penalty}_leehon_confusion.npy\
  --input_ids $phn_ids\
  --leehon_mapping conf/phones.48-39\
  --phn_ids $exp_dir/svm_train_all_pairs_${penalty}_phns.ids\
  --top_predicted_labels $exp_dir_bernoulli/bernoulli_dev_top_predicted_labels.npy

#done

\end{verbatim}
and this proposed ``fix'' only gets us 70\%
accuracy which is a step in the wrong direction from the SVM, although it is an improvement over the Bernoulli model.
Instead we are going to use the Bernoulli mixture modeling in order
to cluster the instances and figure out where the mistakes are being
made.

This gets into a hierarchical learning method where we train
true detections for a cluster against its mistakes.  In this
setup we associate each data point with its predicted class
and the predicted label.  We run a command which goes through
every class, and every components, finds the hits, mistakes, and
misses, and then we construct from that a training set which
may then be used to train an SVM. 

The first attempt is to do a one-vs-rest strategy, in which case
we train an SVM for each component.  We will then pick the component that
achieved the largest likelihood and is classified as yes.
The code for running the SVMs is here
\begin{verbatim}
data=data/local/data
exp_dir=exp/all_phones_exp
exp_dir_bernoulli=exp/bernoulli_all_phones
dev_examples=""
train_examples=""
train_lengths=""
dev_lengths=""
phn_ids=""
phn_id=0
phns=`cat $data/phone.list`
for phn in `cat $data/phone.list` ; do
  dev_examples="$dev_examples data/local/data/${phn}_dev_examples.npy"
  dev_lengths="$dev_lengths data/local/data/${phn}_dev.lengths"
  train_examples="$train_examples data/local/data/${phn}_train_examples.npy"
  train_lengths="$train_lengths data/local/data/${phn}_train.lengths"
  phn_ids="$phn_ids $phn_id"
  phn_id=$(( $phn_id + 1 ))
done



python local/CTrainComponentSVMs.py --data $train_examples \
    --labels ${exp_dir_bernoulli}/bernoulli_train_labels.npy\
    --components ${exp_dir_bernoulli}/bernoulli_train_components.npy\
    --predicted_labels ${exp_dir_bernoulli}/bernoulli_train_top_predicted_labels.npy\
    --predicted_components ${exp_dir_bernoulli}/bernoulli_train_top_predicted_components.npy\
   -v\
    --config conf/main.config\
    --label 0\
    --component_id 0\
    --out_svm_prefix ${exp_dir_bernoulli}/svm_bernoulli_train_0l_0c_



\end{verbatim}
and having established that works we can then run a multicore version
training all of them at once. We do this by first generating a list
of all the labels and components
\begin{verbatim}
commands_fl=${exp_dir_bernoulli}/commands_fl
rm -f $commands_fl
touch $commands_fl
python local/CGetAllLabels.py --labels ${exp_dir_bernoulli}/bernoulli_train_labels.npy | while read label ; do
    python local/CGetAllComponents.py --labels ${exp_dir_bernoulli}/bernoulli_train_labels.npy --components ${exp_dir_bernoulli}/bernoulli_train_components.npy --label $label | while read component ; do
        echo python local/CTrainComponentSVMs.py --data $train_examples \
    --labels ${exp_dir_bernoulli}/bernoulli_train_labels.npy\
    --components ${exp_dir_bernoulli}/bernoulli_train_components.npy\
    --predicted_labels ${exp_dir_bernoulli}/bernoulli_train_top_predicted_labels.npy\
    --predicted_components ${exp_dir_bernoulli}/bernoulli_train_top_predicted_components.npy\
   -v\
    --config conf/main.config\
    --label $label\
    --component_id $component\
    --out_svm_prefix ${exp_dir_bernoulli}/svm_bernoulli_train_${label}l_${component}c_ >> $commands_fl
    done
done

cat $commands_fl | while read line; do
sem --jobs 6 $line
done
\end{verbatim}
and then we have the set of SVMs that we are going to use
to perform testing with.  The first thing to check now is the testing
loss given that we have the new classifier on hand.

\begin{verbatim}

\end{verbatim}

  In particular, we note that the true label
is often among the top five predictions.  So we are going to
rerun the SVM but we only hold the voting among the top five
predicted classes using the Bernoulli mixture model.



\section{Latent Shift EM for Detection performance}
Another experiment we are running is to compare detection results where we double the length of the examples used for detection.
To dod this we construct another config file with a longer time
\texttt{conf/main\_longer.config}
which we get via
\begin{verbatim}
cat << "EOF" > $conf/main.config
[SPECTROGRAM]
sample_rate=16000
num_window_samples=320
num_window_step_samples=80
fft_length=512
kernel_length=7
freq_cutoff=3000
preemphasis=.95
use_mel=False
do_mfccs=False
no_use_dpss=False
mel_nbands=40
num_ceps=13
liftering=.6
include_energy=False
include_deltas=False
include_double_deltas=False
delta_window=9
do_freq_smoothing=False
mel_smoothing_kernel=-1

[SVM]
example_length=.5
kernel='linear'
penalty_list=['unreg', '1.0',
                                                 'little_reg','0.1',
                                                 'reg', '0.05',
                                                 'reg_plus', '0.01',
                                                 'reg_plus_plus','0.001']


[EDGES]
block_length=40
spread_length=1
threshold=.7
magnitude_block_length=0
abst_threshold=(0.025,  0.025,  0.015,  0.015,  0.02 ,  0.02 ,  0.02 ,  0.02 )
magnitude_threshold=.4
magnitude_spread=1
magnitude_and_edge_features=False
magnitude_features=False
mag_smooth_freq=0
mag_downsample_freq=0
auxiliary_data=False
auxiliary_threshold=.5
num_mag_channels=10
num_axiliary_data=3
save_parts_S=False
EOF

\end{verbatim}
so that the features used are half a second in length. The experiment
to be performed is comparing p versus b so we re extract the
data
\begin{verbatim}
local=local
conf=conf
data=data/local/data
for phn in b p  ; do
    echo ${phn}
    for data_set in train dev core_test ; do
        echo ${data_set}

        $local/sampletrans2frametrans.py -c $conf/main_longer.config\
            --transcript ${data}/${phn}_${data_set}.trans\
            > ${data}/${phn}_${data_set}_longer.frame_trans

        $local/extract_data_matrix.py -c $conf/main_longer.config --transcript ${data}/${phn}_${data_set}_longer.frame_trans\
            --uttids ${data}/${data_set}.uttids\
            --output ${data}/${phn}_${data_set}_longer_examples.npy\
            --output_spec ${data}/${phn}_${data_set}_longer_examples_S.npy\
            --output_info ${data}/${phn}_${data_set}_longer_info.npy
    done
done

\end{verbatim}
and then we train the SVMs to compare these
\begin{verbatim}
export PATH=$PATH:/home/mark/Template-Speech-Recognition/Development/051913
phn1=b
phn2=p
data=data/local/data
datadir=$data
exp_dir=exp/longer_classify
mkdir -p $exp_dir
conf=conf
CSVMTrain.py --input_data_matrices\
            $datadir/${phn1}_train_longer_examples.npy\
            $datadir/${phn2}_train_longer_examples.npy\
            --input_test_data_matrices\
            $datadir/${phn1}_dev_longer_examples.npy\
            $datadir/${phn2}_dev_longer_examples.npy\
            --input_lengths\
            $datadir/${phn1}_train.lengths\
            $datadir/${phn2}_train.lengths\
            --input_test_lengths\
            $datadir/${phn1}_dev.lengths\
            $datadir/${phn2}_dev.lengths\
            --output_fls_prefix $exp_dir/svm_train_longer_${phn1}_${phn2}\
            --config $conf/main_longer.config -v
\end{verbatim}
one question that we wish to address is how much difference the longer
or shorter windows really makes on classification performance.
We use 97 time points for this experiment.
We have a local script for comparing the sizes of the various
coefficient filters


\section{Collecting terms and training}

The first task is to get the set of keywords and how many there are.
Since we are working with switchboard, and, in particular, the
\texttt{swb\_ms98} word transcriptions.  So we build a dictionary of
the words that maps each word to a transcription.
We will create a file \texttt{term\_identifiers} which is of the form
\begin{verbatim}
<term> <term-id>
\end{verbatim}
which will be produced by the script
\begin{verbatim}
mkdir -p data/local
local/extract_words.py -d swb_ms98_transcriptions/sw-ms98-dict.text -o data/local/words.txt
\end{verbatim}
and a file \texttt{data/local/words.txt}
\begin{verbatim}
<word-id> <word>
\end{verbatim}
and then we need to get the counts of these. To do this we first
compile a list of the word transcript files
\begin{verbatim}
cat swb_ms98_transcriptions/*/*/*-word.text | local/collect_word_stats.py -w data/local/words.txt > data/local/words_occurrences_av_durations.txt
\end{verbatim}
and then we want to get the top words by occurrence that are
of short duration
\begin{verbatim}

\end{verbatim}
 The counts are computed
using the script
\begin{verbatim}

\end{verbatim}
which will be const


Our code is significantly different than from before.  One of the files is \texttt{CExtractInstances.py}
which has the task of determining where the instances of a phone are within the data. In particular datum $i$ comes as 
a signal $x_i(n)\in\mathbb{R}$ with $n\in[N_i]=\{0,1,\ldots,N_i-1\}$ paired with a transcript consisting in start times $t_i^{start}(m)\in[N_i]$,
end times $t_i^{end}(m)\in[N_i]$ and labels $p_i(m)\in [P]$ where $m\in[M_i]$ so that $N_i$ is the number of samples (assumed to 
uniformly sample a continuous signal) and $M_i$ is the number of labels.  The transcript forms a \textit{segmentation} of
$[N_i]$ so that $t_i^{start}(m+1)=t_i^{end}(m)$ so that the labels are a division of the samples.  In general $M_i << N_i$.

Our task is to produce a set of start times $\{\hat{t_k}\}_{k=0}^{\hat{K}-1}$ for a query label symbol $p\in[P]$ on an unlabeled
signal $x$.  The algorithm we use for this task may be trained with a data set $\{(x_i,t_i^{start},t_i^{end},p_i)\}_{i=0}^{I-1}$.  The 
approach we take is to compute a generalized log-likelihood ratio test $\lambda(n';\Theta_p,\theta_{bgd})$ at each time point
$t$ so that
\begin{equation}
  \lambda(n';x,\Theta_p,\theta_{bgd}) = \max_{\theta\in\Theta_p}\log\frac{\mathbb{P}(x[n-n'];\theta)}{\mathbb{P}(x[n-n'];\theta_{bgd})}
\end{equation}
where $\Theta_p$ is a set of models for the signal if the signal label is $p$ (starting at sample $n=0$).
The curve produced by $\lambda(n'; x,\Theta_p,\theta_{bgd})$ for all values $n'$ is then reduced to a finite set of times $\{\hat{t_k}\}_{k=0}^{\hat{K}-1}$ using a peak-finding algorithm.  We also include an optional peak-clustering step where the peaks are partitioned
into contiguous time intervals (or clusters).  For scoring purposes we label each peak (or peak cluster) as being a \textit{hit}
or a \textit{false alarm}.  In the case of peaks we consider the cluster a hit if the peak time is close to a labeled start time
for the query label $p$. A peak cluster is a hit if the labeled start time occurs within the cluster.  A peak or a peak cluster
is a false alarm otherwise.  We measure performace with an ROC curve which tracks the \textit{true positive rate}, or the fraction of
true labeled start times that our detector finds and compares it to the \textit{false positive rate}, or the number of false alarms
per hour (or other unit of time).

\section{Building the Model}

To build the model we extract all instances within a subset of the data.

\subsection{Extracting the Data}

We use several tools for extracting the data. The first is to just get the list of samples and locations for where the 
target phone sequence occurs--that task is handled by \texttt{CObjectInstances.py}. It takes as an argument a label sequence, a config file
(which wil be standard throughout the system), and a two column list of transcript-audio file path pairs. The command is invoked
\begin{verbatim}
paste train.phn train.wav | ./CObjectInstances.py \
    --label_sequence $PHNSEQUENCE  > extracted_instances
\end{verbatim}
the output is a text file where each line is given by the following rule:
\begin{verbatim}
<file-path> <object-start-sample> <object-end-sample> \
    (| <object-start-sample> <object-end-sample>)*
\end{verbatim}
so that we can have multiple instances.  However, for performing model inference and such we want to also extract some object context
since we wish to estimate the detection thresholds that will be used to find objects.

From here we could just extract the objects exactly at
their labeled samples.  This approach is deficient, however,
because of the way we estimate our statistical models and the
way we perform detection because the detector will not generally
fire exactly at the labeled start time for the object.  Thus
associated with the object $i$ start sample, $t_i^{start}$, is a detection
radius $r_i^{detect}$ so that a detection firing at a sample
$t$ with $|t-t_i^{start}|<r_i^{detect}$ is considered a \textit{hit}.
In order to estimate the behavior of a particular detection
algorthm
we want to extract some context around the object, so that
we can compute the detectition statistics $\Lambda(t)$ at least
for all times in $B(t_i^{start},r_i^{detect})$.  Moreover,
our detector works as a sliding-window of fixed size
(or potentially multiple such windows in parallel) where the size
of the sliding window is chosen in a data-driven manner so the
estimated sliding-window might be longer than the object we are
testing against so we want to be sure to extract enough surrounding
context of the object that we can test a sliding window of any
potential length against any object.  The only bound we have
on the length of the detection window is the length of the
labeled examples $l_{\max}$.  The minimal segment of features
that we want to extract will then be
$$[t_i^{start}-r_i^{detect},t_i^{start}+r_i^{detect}+l_{\max}]$$
and we can get $l_{\max}$ from the \texttt{object\_instances} file,
but we also need to know $r_i^{detect}$, which we just set to be
$l_i/3$ where $l_i$ is the object length.

Different options could be set for these things and we are
attempting to write code so that way these are handled elegantly
and quickly.  These options are handled in main.config:
\begin{verbatim}
[DETECTION]
adaptive_window=True
window_radius=.3
\end{verbatim}
where this indicates that the radius of the window is to be $r_i=.3\cdot l_i$
where $l_i$ is the length of the example.  A window with that radius will have
length $2*r_i+1$.
If we were to use the same radius $r$ for each window on each example the config
file would have
\begin{verbatim}
[DETECTION]
adaptive_window=False
window_radius=10
\end{verbatim}
to indicate that we want a fixed window radius around all of the
examples.  The tool we use is \begin{verbatim}
CObjectContextSegments.py\end{verbatim}
and we run the command 
\begin{verbatim}
./CObjectContextSegments.py -f object_instances -c main.config > context_object_instances
\end{verbatim}

The format of \texttt{context\_object\_instances} is
\begin{verbatim}
<file-path> <object-context-times> ( | <object-context-times>)* ...
\end{verbatim}
where
\begin{verbatim}
<object-context-times> = <context-start> <context-end> <object-start> <object-end>
\end{verbatim}
indicates where the object is and what context to extract. We then
construct a file that parses this along with the configuration
file to produce a data set.

In the next step we actually extract the training data. The command
we use is \texttt{CExtractObjectContext.py} it takes the file
\texttt{context\_object\_instances} as input as well as
an argument to decide what to name the output files, an argument
to determine whether to save the waveforms, and a path to a config
file which determines the parameters for computing the Fourier
transform with the file.  The command is run
\begin{verbatim}
./CExtractObjectContext.py -f context_object_instances -o out --do_wave_output -c main.config
\end{verbatim}
and then we can listen to the output and see the pictures of
the output as well.
\begin{verbatim}

\end{verbatim}

  The output files from that command
run are
\begin{verbatim}
out_X.npy   # the binary edge features for the training examples
out_S.npy   # spectrogram features for the training examples
out_W.npy   # concatenated waveforms
out_meta.txt  # text file containing when things happen
out_bgd_E.npy # the background data file
\end{verbatim}
The next file to be run is the actual training of the data models:
\begin{verbatim}
./CTrainTemplates.py -f out -o orig -c main.config
\end{verbatim}
where the \texttt{-f out} flag indicates the file prefices for the
data that will be used to estimate the templates, 
\texttt{-o orig} is the prefix for the output files,
and \texttt{-c main.config} is for the configuration file.  The 
structure of this file is
\begin{enumerate}
\item Load in data
\item Manipulate edge feature data to prepare for the Bernoulli EM algorithm
\item Perform the Bernoulli estimation
\item Extract the template visualizations
\item Estimate the detection thresholds for all the examples.
\end{enumerate}
The output, specifically, that we need  is a matrix containing
the linear-filters, constant terms, and detection thresholds.  


The step after that is to estimate the model, once that has been
estimated we can then perform the cascaded training and ultimately
do the detection experiment.

 this is an intermediate step before   Doing so allows for straightforward testing with templates of any given length.

\section{Bernoulli Template Clustering}

For this portion we begin constructing things that are going to
be specific to the experiment under consideration
so we create a dictionary to store these files
\begin{verbatim}
mkdir -p exp/model
\end{verbatim}

The command for getting the templates estimated is
\begin{verbatim}
CTrainTemplates.py -f out -o exp/model/templates -c main.config -v
\end{verbatim}
and these are some example templates trained. The output is
the files
\begin{verbatim}
templates_constants.npy
templates_E_filters.npy
templates_affinities.npy
templates_E_templates.npy
templates_S_templates.npy
\end{verbatim}
  The algorithm is simple and just uses
EM to estimate the templates.  The next step is to get the set of filters, which need not all the 
the same length.  The code for constructing the filters is
based on the log-likelihood ratio test.


\section{Patchwork of Parts Model}

To make the problem easier we are going to first cluster the templates using the standard Bernoulli methods
and get the affinities.


\section{Running the Recognizer}



There are a couple of components to this. The first is to generate and template score series for each of the data files, these data
series will be stored in \texttt{exp/train}. We generate them
using the command
\begin{verbatim}
mkdir -p exp/loglike_ratio
CTemplateScore.py -d train.wav -c main.config -t exp/model/templates -o exp/loglike_ratio
\end{verbatim}
and then this will output the score files.
Those files are saved in 
\begin{verbatim}
exp/loglike_ratio/scores_meta.npy
exp/loglike_ratio/scores.npy
exp/loglike_ratio/labels_meta.npy
exp/loglike_ratio/labels.npy
exp/loglike_ratio/utt_ids
\end{verbatim}
where \texttt{scores\_meta.npy} is a two-dimensional array
where each row corresponds to a score series
stacked together in \texttt{scores.npy}, the first entry of the row
is the utterance id in \texttt{utt\_ids}, the second entry
is the mixture component, and the third entry is the number
of frames.  The labels indicate where the keyword occurs.

\section{Inferring Detection Thresholds and Base Detections}

We consider two types of thresholds here. One is the correction
that is done on the scores for off-object this allows us to do
an FDR-style detection procedure for the true positives
in the time series.

\subsection{Likelihood Ratio Test Statistic Distribution Estimation}

The first step is to figure out how the likelihood ratio test
statistic is distributed on the data. And we do this with
\texttt{CGetLRTMeanVar.py}
and we run it with
\begin{verbatim}
CGetLRTMeanVar.py -i exp/loglike_ratio -m exp/model -c main.config
\end{verbatim}
which gets us the mean values and standard deviations on the performance of the
detector on those data.  These can provide a diagnostic to show
that a certain component should not be used. 

\subsection{Detection Threshold}

We first infer the detection threshold that will be used. Detections will correspond to local maxima
of the detector, which may be seen from the detector sequence.  To get these we need
to get the scores and use the meta of the scores to get a detection sequence.  Namely,
we designate a score as a maximum if it is a maximum over all detectors as well as within a small
interval of time.
\begin{verbatim}
CGetDetectThresholds.py -i exp/loglike_ratio -m exp/model -c main.config
\end{verbatim}
the input that this file requires is
\begin{verbatim}
exp/loglike_ratio/scores.npy
exp/loglike_ratio/scores_meta.npy
exp/loglike_ratio/labels.npy
exp/model/templates_E_filters_midpoints.npy
\end{verbatim}
and then we save the results
\begin{verbatim}
exp/loglike_ratio/train_detection_scores.npy
exp/loglike_ratio/train_detection_meta.npy
exp/loglike_ratio/train_hit_scores.npy
exp/loglike_ratio/train_hit_meta.npy
\end{verbatim}
and additionally we get the hits and false alarms
arranged by which component detected them

and these are were the detections are saved.  We also create a file for performing the clustering
of these detection scores.

\section{Training SVMS}

The SVM


\section{Get Detections}

Detections are defined as local maxima that exceed the detection
threshold.
\begin{verbatim}
CGetDetections.py -i exp/loglike_ratio
\end{verbatim}


\section{Detection Clustering}

We now turn the series into clusters.  We use the procedure
in ``Lattice Indexing for Spoken Term Detection''
and we cluster the detections based on end time.
\begin{verbatim}
CClusterDetections.py -i exp/loglike_ratio -m exp/model
\end{verbatim}
based on the paper the steps are as follows:
\begin{enumerate}
\item sort the collected (start, end) times pairs with respect to end times
\item identify the largest set of non-overlapping set of (start, end) times and assign them as cluster heads
\item classify the rest of the pairs according to maximal overlap
\end{enumerate}
We observe that a greedy solution works for the second problem.  Namely
we begin with the first element of the list and then we

To run this algorithm we need to compute the overlap between two
intervals
the proposed function is
\begin{equation}
  overlap((a,b),(c,d)) = \{\min\{b,d\} - \max\{a,c\}\}_+
\end{equation}
since this is only non-zero iff $c<b$ and $a<d$ (as the two other
inequalities are trivially implied by the intervals), which is a 
sufficient condition for overlap.  If they are overlapping then
the intersection between them will be an interval and the end
point of that intersection will be the minimum of the two ends
and the start will be the maximum of the two starts, which gives
us the formula.

The main statistics that we care about in these clusters is the extent of the cluster, the start time and identity of the detection with the largest score (representative mean of the cluster).  That
detection will be the representative cluster that we actually
output and use for scoring.

\section{Setting up the cascade}

In this section rather than just considering the clusterings
\begin{verbatim}
./CGetPosNeg.py -i exp/loglike_ratio -m exp/model
\end{verbatim}



\section{Evaluation}

The standard for evaluation is that a system is output is judged
as correct if the midpoint of the system output occurrence is 
less than or equal to 0.5 seconds from the time span of a known
occurrence of the search.  There is also the constraint a detection
can be counted only for a single example and that an example only one detection can be counted as correct otherwise its a miss.


\section{Detection Thresholds for ROC estimation}

The next step is to find, for each positive training example,
the largest 

The algorithm is conducted in stages. We find all positive
locations and we look within the 
detection window for maximal score within those locations.  

\section{Unsupervised template learning}

We consider the general topic of Unsupervised template learning. First we consider an extension of the CSSR approach as given in
\cite{shalizi04}.  Our setting is that we have Bernoulli vectors computed from several utterances for a single speaker. The code
for this is given in \texttt{timit\_template\_learn\_data\_prep.sh} which shows that this is a TIMIT-focused experiment. Eventually
I will test this on switchboard, etc.

Just for the sake of simplicity I am going to work with a single female TIMIT speaker to get the ball rolling on this thing. First thing
to do is to extract the features.  The choice of speaker is \texttt{fcjf0} whose utterances are in \texttt{train/dr1} the utterances are:
\begin{verbatim}
sa1.wav
sa2.wav
si1027.wav
si1657.wav
si648.wav
sx127.wav
sx217.wav
sx307.wav
sx37.wav
sx397.wav
\end{verbatim}

I shall construct a script that converts these all into a single matrix (that will be divided to get training and development subsets).
I will begin with the HTK feature extraction followed by my own.  I want to have the style of script be similar to what I saw in the 
Kaldi framework, and everything is done with shell commands which have been:
\begin{verbatim}
./timit_template_learn_data_prep.sh
WORKDIR=/home/mark/Research/Sufficiency/TemplateLearning/work
SCRIPTDIR=/home/mark/Research/Sufficiency/TemplateLearning/bin
PATH=$PATH:$HTK/HTKTools:$SCRIPTDIR
DATA=$WORKDIR/data
export PATH
for ml in 1 2 3 4 5 6 7 ; do
CClusterFeatures.py -c $WORKDIR/main.config -t $DATA/speaker_E_train.npy -d $DATA/speaker_E_dev.npy -l $ml -o $WORKDIR/model${ml}.npz --num_mix_set 1 2 4 6 8 16 32 64 128 -ts $DATA/speaker_S_train.npy
done
\end{verbatim}

The data portion of the program is summarized in \texttt{\$DATA/speaker\_meta.txt}
which indicates that five utterances were used for the training
data:
\begin{verbatim}
sa1.wav
sa2.wav
si1027.wav
si1657.wav
si648.wav
\end{verbatim}
and there were a total of 2905 frames of data included.  I used
Bernoulli mixture clustering 
\begin{verbatim}
template_speech_rec.bernoulli_mixture
\end{verbatim}
on the individual frames.  The next step is to estimate
\begin{equation}
  \hat{\mathbb{P}}(S_t\mid S_{t-1}, X_0^{T-1})
\end{equation}
where $S_t$ is the latent variable associated with an observation $X_t$.
The length one model is just a set of models and weights $\{(T_1^i,w_1^i)\}_{i=1}^{k_1}$
and thus we can compute a vector $A_t^i$ which are the normalized likelihoods of
$X_t$ under model $T_1^i$ weighted by $w_1^i$:
\begin{equation}
  A_t^i = \frac{w_1^i\mathbb{P}_{T_1^i}(X_t)}{\sum_{j=1}^{k_1}w_1^j\mathbb{P}_{T_1^j}(X_t)}
\end{equation}
so then we have
\begin{equation}
  \begin{array}{rcl}
  \hat{\mathbb{P}}(S_t\mid S_{t-1}) &=& \frac{\mathbb{P}(S_t, S_{t-1})}{\mathbb{P}(S_{t-1})}\\
  &=& \frac{\sum_{t=1}^{T-1} A_t^i A_{t-1}^j}{\sum_{t=1}^{T-1}A_{t-1}^j}
  \end{array}
\end{equation}
There is some subtlety to this, however, since there is a margin: i.e. points that don't havea complete history on either side. So we need to be really careful about defining these things.  We assume that
we have observations $\{x_0^{T-1}\}$ so that means the history ranges
over $\{x_0^{T-2}\}$ and the future ranges over $\{x_1^{T-1}\}$. Really
what we have here is a matrix $W_{ij}^1$ which maps a state $j$
to a state $i$.  Thus an important thing to use here are the
affinities.  Thus we are in a similar situation to the standard
HMM baum-welch algorithm. Next we want to distinguish
sequences and do comparison for different length-clusterings.
In particular we also have clusters for pairs of observations
$(X_t,X_{t-1})$ and we need to know whether a pair has a different 
predictive future distribution than an independent pair of
observations.  There is some kind of multiple-testing difficulty
that we encounter inherently by doing this and some kind of
false-discovery rate or whatever.

In any case the affinities are the main thing being used to get
these predictions.  Before I continue I want to actually show
what the data has produced. An interesting question is whether
we are picking up any kind of differences between the different
vowels and consonants and what not with this kind of clustering.

 I do this by running the command
\begin{verbatim}
PLOTS=$WORKDIR/plots
mkdir -p $PLOTS
CViewModel -f $WORKDIR/model1.npz -g $PLOTS/model1
\end{verbatim}
 can associated with each $X_t$ 


\end{document}
